% LaTeX Curriculum Vitae Template
%
% Copyright (C) 2004-2009 Jason Blevins <jrblevin@sdf.lonestar.org>
% http://jblevins.org/projects/cv-template/
%
% You may use use this document as a template to create your own CV
% and you may redistribute the source code freely. No attribution is
% required in any resulting documents. I do ask that you please leave
% this notice and the above URL in the source code if you choose to
% redistribute this file.

\documentclass[letterpaper]{article}

\usepackage{hyperref}
\usepackage{geometry}

% Comment the following lines to use the default Computer Modern font
% instead of the Palatino font provided by the mathpazo package.
% Remove the 'osf' bit if you don't like the old style figures.
\usepackage[T1]{fontenc}
\usepackage[sc,osf]{mathpazo}

% Set your name here
\def\name{Romit Maulik}

% Replace this with a link to your CV if you like, or set it empty
% (as in \def\footerlink{}) to remove the link in the footer:
%\def\footerlink{http://jblevins.org/projects/cv-template/}

% The following metadata will show up in the PDF properties
\hypersetup{
  colorlinks = true,
  urlcolor = black,
  pdfauthor = {\name},
  pdfkeywords = {},
  pdftitle = {\name: Curriculum Vitae},
  pdfsubject = {Curriculum Vitae},
  pdfpagemode = UseNone
}

\geometry{
  body={6.5in, 8.5in},
  left=1.0in,
  top=1.25in
}

% Customize page headers
\pagestyle{myheadings}
\markright{\name}
\thispagestyle{empty}

% Custom section fonts
\usepackage{sectsty}
\sectionfont{\rmfamily\mdseries\Large}
\subsectionfont{\rmfamily\mdseries\itshape\large}

% Other possible font commands include:
% \ttfamily for teletype,
% \sffamily for sans serif,
% \bfseries for bold,
% \scshape for small caps,
% \normalsize, \large, \Large, \LARGE sizes.

% Don't indent paragraphs.
\setlength\parindent{0em}

% Make lists without bullets
\renewenvironment{itemize}{
  \begin{list}{}{
    \setlength{\leftmargin}{1.5em}
  }
}{
  \end{list}
}

\begin{document}

% Place name at left
{\huge \name}

% Alternatively, print name centered and bold:
%\centerline{\huge \bf \name}

\vspace{0.25in}

\begin{minipage}{0.45\linewidth}
  \href{http://www.mcs.anl.gov/}{Argonne Leadership Computing Facility} \\
  Building 240, Argonne National Laboratory \\
  9700 Cass Avenue, Lemont, IL 60439
\end{minipage}
\begin{minipage}{0.45\linewidth}
  \begin{tabular}{ll}
    Phone: & (405) 982-0161 \\
    %Fax: &  (919) 962-5678 \\
    Email: & \href{mailto:rmaulik@anl.gov}{\tt rmaulik@anl.gov} \\
    Homepage: & \href{https://romit-maulik.github.io/}{\tt romit-maulik.github.io} \\
  \end{tabular}
\end{minipage}


\section*{Current position}
\begin{itemize}
\item Margaret Butler Postdoctoral Fellow, Argonne National Laboratory, Jun, 2019 - Present
\end{itemize}

\section*{Prior experience}

\begin{itemize}
\item Predoctoral Appointee - MCS, Argonne National Laboratory, Jan, 2019 - May, 2019
\item RA - Computational Fluid Dynamics Laboratory, Oklahoma State University, Jan, 2016 - May, 2019.
\item RA - Computational Biomechanics Laboratory, Oklahoma State University, Aug, 2013 - July, 2015.
\item TA - Mechanical \& Aerospace Engineering, Oklahoma State University, Jan, 2013 - Dec, 2018.
\item Design Engineer - Tata Technologies Limited, Pune, Aug,2012 - Aug,2013.
\end{itemize}

\section*{Academic background}

\begin{itemize}
  \item PhD. Mechanical \& Aerospace Engineering, Oklahoma State University, 2019.
  \item M.S. Mechanical \& Aerospace Engineering, Oklahoma State University, 2015.
  \item B.E. Mechanical Engineering, BIT Mesra - India, 2012. 
\end{itemize}

\section*{Research interests}

Scientific machine learning, high-performance computing, reduced-order modeling, turbulence modeling, numerical methods. 

\section*{Projects}

\begin{enumerate}
    \item \textbf{Romit Maulik} (PI), Turb-Net: Scaleable physics-informed deep learning for turbulence model development, Director's discretionary resource allocation (3 million core hours on Theta), Argonne Leadership Computing Facility, Argonne National Laboratory.
\end{enumerate}

\section*{Publications}

\subsection*{In progress}

\begin{enumerate}

\item \textbf{R. Maulik}, O. San, J. Jacob: Connecting implicit and explicit large eddy simulations of two-dimensional turbulence through machine learning, {\it arXiv preprint : 1901.09329, Under review}.

\item \textbf{R. Maulik}, H. Sharma, S. Patel, B. Lusch, E. Jennings : Accelerating RANS turbulence modeling using potential flow and machine learning, {\it arXiv preprint : 1910.10878, Under review}.

\item J. Choi, S. Robinson, \textbf{R. Maulik}, W. Wehde: What Matters the Most for Individual Disaster Preparedness? Understanding Emergency Preparedness Using Machine Learning, {\it Under review}.

\item S. Renganathan \textbf{R. Maulik}, V. Rao : Machine-Learning for Nonintrusive Model Order Reduction of the Parametric Inviscid Transonic Flow past an airfoil, {\it arXiv preprint : 1911.07943, Under review}.

\item \textbf{R. Maulik}, J. Burby, N. Garland, S. Madireddy, X. Tang, P. Balaprakash: Neural network representability of fully ionized plasma fluid model closures, {\it In preparation}.

\item \textbf{R. Maulik}, B. Lusch, P. Balaprakash: Parameteric surrogate models for the shallow-water equations using convolutional autoencoders and long short-term memory networks, {\it In preparation}.

\item \textbf{R. Maulik}, I. Pan, M. Lachlan: Uncertainty quantification of non-intrusive reduced order models for the inviscid shallow water equations, {\it In preparation}.

\item \textbf{R. Maulik}, A. Rappaport, P. Balaprakash, J. Shadid: Parameteric surrogate models for the tearing mode instability problem in resistive magnetohydrodynamics using long short-term memory networks, {\it In preparation}.

\item \textbf{R. Maulik}, V. Rao, B. Lusch, P. Balaprakash: Stable non-intrusive ROMS using non-autoregressive time series methods, {\it In preparation}.

\item S. Renganathan \textbf{R. Maulik}, V. Rao : Aerodynamic Data Fusion using scalable Bayesian Inference, {\it In preparation}.

\item B. Narayanan, \textbf{R. Maulik}, M. Zhou, H. Doan, P. Balaprakash, L. A. Curtiss, R. S. Assary: Protonation of Bio-oil Components from Accurate First principles Simulations and Graph-based neural networks, {\it In preparation}.

\item S. Haering \textbf{R. Maulik} : A machine-learned eddy-viscosity for modeling sub-grid stress and energy transfer in large eddy simulations of channel flow, {\it In preparation}.

% \item \textbf{R. Maulik}, V. Rao, E. Constantinescu, B. Lusch, P. Balaprakash: Using recurrent neural networks for nonlinear component computation in the model-order reduction of the inviscid shallow-water equations, {\it In preparation}.


\end{enumerate}

\subsection*{Peer-reviewed articles}

\begin{enumerate}

\item \textbf{R. Maulik}, A. Mohan, B. Lusch, S. Madireddy, P. Balaprakash, D. Livescu: Time-series learning of latent-space dynamics for reduced-order model closure, {\it Accepted in-press, Physica D., arXiv preprint : 1906.07815}.

\item \textbf{R. Maulik}, R. S. Assary, P. Balaprakash: Site-specific graph neural network for predicting protonation energy of oxygenate molecules, {\it Machine learning for Physical Sciences workshop, NeurIPS}, 2019.

\item \textbf{R. Maulik}, V.Rao, S. Madireddy, B. Lusch, P. Balaprakash: Using recurrent neural networks for nonlinear component computation in advection-dominated reduced-order models, \textit{Machine learning for Physical Sciences workshop, NeurIPS}, 2019.

% \item \textbf{R. Maulik}, O. San: Numerical assessments of a parametric implicit large eddy simulation model, {\it Journal of Computational and Applied Mathematics} (forthcoming).

\item \textbf{R. Maulik}, O. San, J. Jacob, C. Crick: Online turbulence model classification for large eddy simulation using deep learning, {\it Journal of Fluid Mechanics}, 870, 784-812, 2019.

\item O.San, \textbf{R. Maulik}, M. Ahmed: An artificial neural network framework for reduced order modeling of transient flows, {\it Communications in Nonlinear Science and Numerical Simulation}, 77, 271-287, 2019.

\item \textbf{R. Maulik}, O. San, A. Rasheed, P. Vedula: Subgrid modeling for two-dimensional turbulence using artificial neural networks, {\it Journal of Fluid Mechanics}, 858, 122-144, 2019.

\item \textbf{R. Maulik}, O. San, A. Rasheed, P. Vedula: Data-driven deconvolution for large eddy simulation of Kraichnan turbulence, {\it Physics of Fluids}, 30, 125109, 2018.

\item O.San, \textbf{R.Maulik}: Stratified Kelvin-Helmholtz turbulence of compressible shear flows, {\it Nonlinear Processes in Geophysics}, 25, 457--476, 2018.
 
\item O.San, \textbf{R.Maulik}: Extreme learning machine for reduced order modeling of turbulent geophysical flows, {\it Physical Review E}, 97, 042322, 2018.

\item O.San, \textbf{R.Maulik}: Machine learning closures for model order reduction of thermal fluids,  {\it Applied Mathematical Modelling}, https://doi.org/10.1016/j.apm.2018.03.037, 2018.

\item \textbf{R.Maulik}, O.San, R. Behera : An adaptive multilevel wavelet framework for scale-selective WENO reconstruction schemes, {\it International Journal of Numerical Methods in Fluids},\\ https://doi.org/10.1002/fld.4489, 2018.

\item O.San, \textbf{R.Maulik}: Neural network closure models for nonlinear model order reduction, {\it Advances in Computational Mathematics}, https://doi.org/10.1007/s1044, 2018.

\item \textbf{R.Maulik}, O. San: A dynamic closure modeling framework for large eddy simulation using approximate deconvolution: Burgers equation, {\it Cogent Physics}, 5, 1464368, 2018.

\item \textbf{R.Maulik}, O.San: A neural network approach for the blind deconvolution of turbulent flows, {\it Journal of Fluid Mechanics}, 831, 151-181, 2017.

\item \textbf{R.Maulik}, O.San: A novel dynamic framework for subgrid-scale parametrization of mesoscale eddies in quasigeostrophic turbulent flows, {\it Computers and Mathematics with Applications}, 74, 420-445, 2017.

\item \textbf{R.Maulik}, O.San: Explicit and implicit LES closures for Burgers turbulence, {\it Journal of Computational and Applied Mathematics}, 327, 12-40, 2017.

\item \textbf{R.Maulik}, O.San: Resolution and Energy Dissipation Characteristics of Implicit LES and Explicit Filtering Models for Compressible Turbulence, {\it Fluids}, 2(2)-14, 2017.

\item \textbf{R.Maulik}, O.San: A dynamic subgrid-scale modeling framework for Boussinesq turbulence, {\it International Journal of Heat and Mass Transfer}, 108, 1656-1675, 2017.

\item \textbf{R.Maulik}, O.San: A dynamic framework for scale-aware parameterizations of eddy viscosity coefficient in two-dimensional turbulence, {\it International Journal of Computational Fluid Dynamics}, 31(2), 69-92, 2017.

\item \textbf{R.Maulik}, O.San: A stable and scale-aware dynamic modeling framework for subgrid-scale parameterizations of two-dimensional turbulence, {\it Computers \& Fluids} 158, 11-38, 2016.

\item \textbf{R.Maulik}, O.San: Dynamic modeling of the horizontal eddy viscosity coefficient for quasigeostrophic ocean circulation problems, {\it Journal of Ocean Engineering and Science} 1, 300-324, 2016.

\item H. H. Marbini, \textbf{R. Maulik}: A biphasic transversely isotropic poroviscoelastic model for the unconfined compression of hydrated soft tissue, {\it Journal of Biomechanical Engineering} 138, 031003, 2016.

\end{enumerate}

\subsection*{Workshop participation}

\begin{enumerate}
	  \item Invited participant, Algorithms for Dimension and Complexity Reduction, ICERM, Brown University, RI, 2020.

    \item Invited participant, Mathematics of Reduced Order Models, ICERM, Brown University, RI, 2020.

    \item Invited participant, IPAM Workshop III: Validation and Guarantees in Learning Physical Models: from Patterns to Governing Equations to Laws of Nature, UC Los Angeles, October 2019.
    
    \item Invited participant, Department of Energy - AI for Science Townhall, Argonne National Laboratory, June 2019.
    
    \item Invited participant, Advances in PDEs: Theory, Computation and Application to CFD, ICERM, Brown University, RI, 2018.
    \item Invited participant, SDSC Summer program in HPC and Data Science, UC San Diego, 2017.
\end{enumerate}

\subsection*{Contributed talks}

\begin{enumerate}

\item \textbf{R. Maulik}, S. Madireddy, B. Lusch, P. Balaprakash: Closures for Parameteric Reduced-Order Models Using Convolutional Autoencoders, SIAM Conference on Mathematics of Data Science 2020, Cincinnati, Ohio.

\item \textbf{R. Maulik}, H. Sharma, S. Haering, S. Patel, B. Lusch, E. Jennings, P. Balaprakash: General purpose data science for general purpose CFD: Integrating Tensorflow into OpenFOAM at scale, \textbf{Invited talk} at the US-Japan Workshop on Data-Driven Fluid Dynamics, Kobe, Japan, March 2020.

\item J. Choi, S. Robinson, \textbf{R. Maulik}, W. Wehde: What Matters the Most for Individual Disaster Preparedness? Understanding Emergency Preparedness Using Machine Learning, SPSA Conference on Politics of Disasters, Resilience, and Recovery, San Juan, Puerto Rico, 2020.

\item \textbf{R. Maulik}, A. Mohan, S. Madireddy, B. Lusch, P. Balaprakash, D. Livescu: Machine learning of sequential data for non-intrusive reduced-order models, Bulletin of the American Physical Society 72, 2019.

\item \textbf{R. Maulik}, B. Lusch, P. Balaprakash: Tackling the limitations of conventional ROMs for advection-dominated nonlinear dynamical systems using machine learning, \textbf{Invited talk} at the Advanced Statistics meets Machine Learning-III workshop, Argonne National Laboratory, November 2019.

\item \textbf{R. Maulik}, B. Lusch, O. San, P. Balaprakash: Data-driven sub-grid models for the large-eddy simulation of turbulence, \textbf{Invited talk} at John Zink Hamworthy Combustion, Tulsa, 2019.

\item \textbf{R. Maulik}, H. Sharma, S. Patel, E. Jennings, B. Lusch, P. Balaprakash, V. Vishwanath: Novel turbulence closures using physics-informed machine learning, \textbf{Invited talk} at the Argonne Physical Sciences and Engineering Division AI Townhall 2019. 

\item \textbf{R.Maulik}, O.San, A. Rasheed, P. Vedula: Data-driven deconvolution for the sub-grid modeling of large eddy simulations of two-dimensional turbulence, SIAM-CSE, 2019.

\item \textbf{R.Maulik}, O.San, A. Rasheed, P. Vedula: Data-driven deconvolution for the large eddy simulation of Kraichnan turbulence, Bulletin of the American Physical Society 71, 2018.

\item \textbf{R.Maulik}, O.San, C. Bach: A computational investigation of the effect of ground clearance in vertical ducting systems, 2018, Purdue University, Herrick Labs Conferences 2018. 

\item \textbf{R.Maulik}, O.San: A neural network approach for the blind deconvolution of turbulent flows, Bulletin of the American Physical Society 70, 2017.

\item \textbf{R.Maulik}, Ratikanta Behera, O.San: A generalized wavelet based grid-adaptive and scale-selective
implementation of WENO schemes for conservation laws, Texas Applied Mathematics and Engineering Symposium 2017, The University of Texas, Austin.

\item \textbf{R.Maulik}, O.San: An explicit filtering framework based on Perona-Malik anisotropic diffusion for shock capturing and subgrid scale modeling of Burgers' turbulence,  Bulletin of the American Physical Society 69, 2016.

\item \textbf{R.Maulik}, O.San:  A dynamic hybrid subgrid-scale modeling framework for large eddy simulations,  Bulletin of the American Physical Society 69, 2016.

\item O.San, \textbf{R.Maulik}: A dynamic framework for subgrid-scale parametrization of mesoscale eddies in geophysical flows, Bulletin of the American Physical Society 69, 2016.

\end{enumerate}

\section*{Honors \& awards}

\begin{itemize}

\item 3\textsuperscript{rd} Margaret Butler Fellow, Argonne Leadership Computing Facility, Argonne National Laboratory.

\item Best oral presentation, 2\textsuperscript{nd} MAE Graduate Research Symposium, Oklahoma State University, 2018

\item SIAM Student Travel Award: 2019 SIAM Conference on Computational Science and Engineering, Spokane, WA, 2019.

\item Outstanding Graduate Student, College of Engineering Architecture and Technology, Oklahoma State University, 2018

\item Graduate College Robberson Summer Research Fellowship, Oklahoma State University, 2017

\item FGSA Travel Award for Excellence in Graduate Research, American Physical Society, 2017

\item SIAM TX-LA Travel Grant, Texas Applied Mathematics and Engineering Symposium, 2017

\item Graduate Student Travel Grant, American Physical Society - Division of Fluid Dynamics, 2017

\item Graduate Student Travel Grant, Graduate Program Student Government Authority, Oklahoma State University, 2017

\item John Brammer Fellowship, Oklahoma State University, 2016

\item Graduate College Top Tier Fellowship, Oklahoma State University, 2016

\item 7\textsuperscript{th} place in Worldsteel-SteelChallenge 8 - North American University Category, 2014.
\end{itemize}

\section*{Professional service \& outreach}

\begin{itemize}
    \item Reviewer: AIAA Journal, Applied Mathematical Modeling, Computer Physics Communcations, International Journal of Computational Fluid Dynamics, Journal of Fluid Mechanics, Physics of Fluids, Physica D, International Journal of Numerical Methods in Fluids, Nature Communications, Theoretical and Computational Fluid Dynamics.
    \item Program Committee - Argonne Data Science Project allocations : Reviewed allocation requests for 21 million core hours and 190 TB storage.
    \item Co-organizer - Argonne National Laboratory - AI, Statistics and Machine Learning Journal Club.
    \item Tutorial lead - Statistical methods for machine learning, ALCF AI4Science tutorial 2019, Argonne National Laboratory.
    \item Tutorial lead - DeepHyper for scalable hyperparameter and neural architecture search on ALCF machines, ALCF Simulation Data and Learning workshop 2019, Argonne National Laboratory.
    \item Session chair - SIAM Conference on Mathematics of Data Science, Cincinnati, OH, 2020.
    \item Session chair - SIAM Conference on Computational Science and Engineering, Spokane, WA, 2019.
    \item Session chair - 2\textsuperscript{nd} MAE Graduate Research Symposium, Oklahoma State University, 2018.
    \item Organizer \& Lecturer - 2-day TensorFlow workshop, Mechanical \& Aerospace Engineering, 2018
    \item Organizer - National Lab Day outreach, Computational Fluid Dynamics Laboratory, Oklahoma State University 2017, 2018.
\end{itemize}

\bigskip

% Footer
%\begin{center}
%  \begin{footnotesize}
%    Last updated: \today \\
%    \href{\footerlink}{\texttt{\footerlink}}
%  \end{footnotesize}
%\end{center}

\end{document}
