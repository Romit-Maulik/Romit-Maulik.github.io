% LaTeX Curriculum Vitae Template
%
% Copyright (C) 2004-2009 Jason Blevins <jrblevin@sdf.lonestar.org>
% http://jblevins.org/projects/cv-template/
%
% You may use use this document as a template to create your own CV
% and you may redistribute the source code freely. No attribution is
% required in any resulting documents. I do ask that you please leave
% this notice and the above URL in the source code if you choose to
% redistribute this file.

\documentclass[letterpaper]{article}

\usepackage{hyperref}
\usepackage{geometry}

% Comment the following lines to use the default Computer Modern font
% instead of the Palatino font provided by the mathpazo package.
% Remove the 'osf' bit if you don't like the old style figures.
\usepackage[T1]{fontenc}
\usepackage[sc,osf]{mathpazo}

% Set your name here
\def\name{Romit Maulik}

% Replace this with a link to your CV if you like, or set it empty
% (as in \def\footerlink{}) to remove the link in the footer:
%\def\footerlink{http://jblevins.org/projects/cv-template/}

% The following metadata will show up in the PDF properties
\hypersetup{
  colorlinks = true,
  urlcolor = black,
  pdfauthor = {\name},
  pdfkeywords = {},
  pdftitle = {\name: Curriculum Vitae},
  pdfsubject = {Curriculum Vitae},
  pdfpagemode = UseNone
}

\geometry{
  body={6.5in, 8.5in},
  left=1.0in,
  top=1.25in
}

% Customize page headers
\pagestyle{myheadings}
\markright{\name}
\thispagestyle{empty}

% Custom section fonts
\usepackage{sectsty}
\sectionfont{\rmfamily\mdseries\Large}
\subsectionfont{\rmfamily\mdseries\itshape\large}

% Other possible font commands include:
% \ttfamily for teletype,
% \sffamily for sans serif,
% \bfseries for bold,
% \scshape for small caps,
% \normalsize, \large, \Large, \LARGE sizes.

% Don't indent paragraphs.
\setlength\parindent{0em}

% Make lists without bullets
\renewenvironment{itemize}{
  \begin{list}{}{
    \setlength{\leftmargin}{1.5em}
  }
}{
  \end{list}
}

\begin{document}

% Place name at left
{\huge \name}

% Alternatively, print name centered and bold:
%\centerline{\huge \bf \name}

\vspace{0.25in}

\begin{minipage}{0.45\linewidth}
  Mathematics and Computer Science Division \\
  Building 240, Argonne National Laboratory \\
  9700 Cass Avenue, Lemont, IL 60439
\end{minipage}
\begin{minipage}{0.45\linewidth}
  \begin{tabular}{ll}
    Phone: & (405) 982-0161 \\
    %Fax: &  (919) 962-5678 \\
    Email: & \href{mailto:rmaulik@anl.gov}{\tt rmaulik@anl.gov} \\
    Homepage: & \href{https://romit-maulik.github.io/}{\tt romit-maulik.github.io} \\
  \end{tabular}
\end{minipage}

\section*{Research interests}

Scientific machine learning, high-performance computing, reduced-order modeling, numerical methods, stochastic processes, fluid dynamics, geophysical sciences.

\section*{Education}

\begin{itemize}
  \item PhD. Mechanical \& Aerospace Engineering, Oklahoma State University. \hfill \textbf{2016-2019}
  \item M.S. Mechanical \& Aerospace Engineering, Oklahoma State University. \hfill \textbf{2013-2015}
  \item B.E. Mechanical Engineering, Birla Institute of Technology, India. \hfill \textbf{2008-2012}
\end{itemize}


\section*{Positions held}
\begin{itemize}

\item  \textbf{Jun, 2021 - Present} \\ Assistant Computational Scientist, Mathematics and Computer Science, Argonne National Laboratory.

\item \textbf{Oct, 2020 - Present} \\ Research Assistant Professor, Department of Applied Mathematics, IIT-Chicago.

\item \textbf{Jun, 2019 - May, 2021} \\ Margaret Butler Postdoctoral Fellow, Leadership Computing Facility, Argonne National Laboratory.

\item \textbf{Jan, 2019 - May, 2019} \\ Predoctoral Appointee - Maths and Computer Science Division, Argonne National Laboratory.  

\item \textbf{Jan, 2016 - Jan, 2019} \\ Research Assistant - Computational Fluid Dynamics Laboratory, Oklahoma State University. 

\item \textbf{Aug, 2013 - Jul, 2015} \\ Research Assistant - Computational Biomechanics Laboratory, Oklahoma State University. 

\item  \textbf{Jan, 2013 - Dec, 2018} \\ Teaching Assistant (Introductory Dynamics, Introductory Fluid Mechanics, Practical CFD) - Mechanical \& Aerospace Engineering, Oklahoma State University.

\item \textbf{Aug, 2012 - Aug, 2013} \\ Design Engineer - Tata Technologies Limited, India. 
\end{itemize}

\section*{Honors \& awards}

\begin{itemize}

\item \href{https://www.alcf.anl.gov/margaret-butler-fellowship-computational-science}{Margaret Butler Fellow}, Argonne Leadership Computing Facility, Argonne National Laboratory, 2019-2021.

\item SIAM Travel Grant: 2019 SIAM Conference on Computational Science and Engineering, Spokane, WA, 2019.

\item Outstanding Graduate Student, College of Engineering Architecture and Technology, Oklahoma State University, 2018.

\item Graduate College Robberson Summer Research Fellowship, Oklahoma State University, 2018.

\item SIAM TX-LA Section Travel Grant, Texas Applied Mathematics and Engineering Symposium, 2017.

\item Graduate Student Travel Grant, American Physical Society - Division of Fluid Dynamics, 2017.

\item Graduate Student Travel Grant, Graduate Program Student Government Authority, Oklahoma State University, 2017.

\item FGSA Travel Grant for Excellence in Graduate Research, American Physical Society, 2017.

\item John Brammer Fellowship, Oklahoma State University, 2016.

\item Graduate College Top Tier Fellowship, Oklahoma State University, 2016.

\end{itemize}

\section*{Committee membership}

\begin{itemize}
\item DOE INCITE program (2020) - Reviewed 2 proposals every year
\item ADSP program (2020) - Reviewed 2 proposals every year
\item International Conference on Parallel Processing, Chicago, 2021 (Reviewed 6 articles).
\item Wilkinson Postdoctoral Fellowship Committee, MCS Division, Argonne National Laboratory, 2022.
\end{itemize}

\section*{Supervising}

\begin{itemize}

  \item Sahil Bhola (University of Michigan): Multifidelity reinforcement learning for computational fluid dynamics, Research Aide, Summer 2021.

  \item Alec Linot (University of Wisconson): Deep learning of dynamical systems on inertial manifolds, Wallace Givens Associate, Summer 2021.

  \item William McClure (IIT-Chicago): Estimating the Generator of SDEs Using Temporal Normalizing Flows, Masters Thesis, 2021. 

  \item Janah Richardson: Causal Relationship Between Environmental Factors and Social Mobility, Argonne mentor for the Afro-Academic, Cultural, Technological and Scientific Olympics (ACT-SO) High School Research Program, 2020-2021. Gold medal winner in Computer Science category- Illinois.

  \item Suraj Pawar (Oklahoma State): Scalable reinforcement learning for computational fluid dynamics, Research Aide, 2020.

  \item Dominic Skinner (MIT): Deep learning reduced-order models for computational physics applications, National Science Foundation, Mathematical Sciences Graduate Internship, Summer 2020.

\end{itemize}


\section*{Publications}

\subsection*{Under review}

\begin{enumerate}

\item \textbf{R. Maulik}, D. Fytanidis, B. Lusch, S. Patel, V. Vishwanath: PythonFOAM: In-situ data analyses with OpenFOAM and Python, {\it arXiv:2103.09389}.

\item J. Choi, W. Wehde, \textbf{R. Maulik}: Comparing Relative Decisiveness for Public Support of Climate Change Mitigation Policies Using Machine Learning.

\item Y. Lu, \textbf{R. Maulik}, T. Gao, F. Dietrich, I. Kevrekidis, J. Duan: Learning the temporal evolution of multivariate densities via normalizing flows, {\it arXiv:2101.00554}.

\item \textbf{R. Maulik}, G. Mengaldo: PyParSVD: A streaming, distributed and randomized singular-value-decomposition library, {\it arXiv:2108.08845}.

\item S. Renganathan, \textbf{R. Maulik}, G. Iungo, S. Letizia: Data-driven wind turbine wake modeling using probabilistic machine learning, {\it arXiv:2109.02411}.

\item G. Iungo, \textbf{R. Maulik}, S. Renganathan, S. Letizia: Machine-learning identification of the variability of mean velocity and turbulence intensity for wakes generated by onshore wind turbines: Cluster analysis of wind LiDAR measurements, {\it arXiv:2109.01646}.

\item M. Morimoto, K. Fukami, \textbf{R. Maulik}, R. Vinuesa, K. Fukagata: Assessments of model-form uncertainty using Gaussian stochastic weight averaging for fluid-flow regression, {\it arXiv:2109.08248}.

\item K. Lyras, \textbf{R. Maulik}, D. Schmidt: Machine-learning accelerated turbulence modelling of transient flashing jets, {\it arXiv:2109.15203}.

\end{enumerate}


\subsection*{Peer-reviewed journal articles}

\begin{enumerate}

\item K. Fukami, \textbf{R. Maulik}, N. Ramachandra, K. Taira, K. Fukagata: Global field reconstruction from sparse sensors with Voronoi tessellation-assisted deep learning, {\it Nature Machine Intelligence, Accepted, arXiv:2101.00554}.

\item B. Hamzi, \textbf{R. Maulik}, H. Owhadi: Simple, low-cost, and accurate, data-driven geophysical forecasting with learned kernels, {\it Proceedings of the Royal Society A}, 477: 20210326, 2021.

\item G. Mengaldo, \textbf{R. Maulik}, PySPOD: A Python package for Spectral Proper Orthogonal Decomposition (SPOD), {\it Journal of Open Source Software}, 6 (60), 2862, 2021.

\item \textbf{R. Maulik}, B. Lusch, P. Balaprakash: Reduced-order modeling of advection-dominated systems with recurrent neural networks and convolutional autoencoders , {\it Physics of Fluids}, 33, 037106, 2021 (Editor's pick).

\item S. Pawar, \textbf{R. Maulik}: Distributed deep reinforcement learning for simulation control, {\it Machine Learning: Science and Technology}, 2, 025029, 2021.

\item S. Renganathan, \textbf{R. Maulik}, J. Ahuja: Enhanced data efficiency using deep neural networks and Gaussian processes for aerodynamic design optimization, {\it Aerospace Science and Technology}, 111, 106522, 2021.

\item J. Burby, Q. Tang, \textbf{R. Maulik}: Fast neural Poincar\'{e} maps for toroidal magnetic fields, {\it Plasma Physics and Controlled Fusion}, 63, 024001, 2021.

\item \textbf{R. Maulik}, T. Botsas, N. Ramachandra, M. Lachlan, I. Pan: Latent-space time evolution of non-intrusive reduced-order models using Gaussian process emulation, {\it Physica D: Nonlinear Phenomena}, 132797, 2021. 

\item \textbf{R. Maulik}, H. Sharma, S. Patel, B. Lusch, E. Jennings : A turbulent eddy-viscosity surrogate modeling framework for Reynolds-Averaged Navier-Stokes simulations, {\it Computers and Fluids}, 104777, 2020. 

\item \textbf{R. Maulik}, K. Fukami, N. Ramachandra, K. Fukagata, K. Taira : Probabilistic neural networks for fluid flow model-order reduction and data recovery, {\it Physical Review Fluids}, 5, 104401, 2020. 

\item \textbf{R. Maulik}, P. Balaprakash, B. Lusch: Non-autoregressive time-series methods for stable parameteric reduced-order models, {\it Physics of Fluids}, 32, 087115, 2020 (Editor's pick). 

\item \textbf{R. Maulik}, N. Garland, X. Tang, P. Balaprakash: Neural network representability of fully ionized plasma fluid model closures, {\it Physics of Plasmas}, 27, 072106, 2020.

\item J. Choi, S. Robinson, \textbf{R. Maulik}, W. Wehde: What Matters the Most for Individual Disaster Preparedness? Understanding Emergency Preparedness Using Machine Learning, {\it Natural Hazards}, 103, 1183-1200, 2020. 

\item S. Renganathan \textbf{R. Maulik}, V. Rao : Machine learning for nonintrusive model order reduction of the parametric inviscid transonic flow past an airfoil, {\it Physics of Fluids}, 32, 047110, 2020. 

\item \textbf{R. Maulik}, O. San: Numerical assessments of a parametric implicit large eddy simulation model, {\it Journal of Computational and Applied Mathematics}, 112866, 2020.

\item \textbf{R. Maulik}, O. San, J. Jacob: Spatiotemporally dynamic implicit large eddy simulation using machine learning classifiers, {\it Physica D: Nonlinear Phenomena}, 406, 132409, 2020. 

\item \textbf{R. Maulik}, A. Mohan, B. Lusch, S. Madireddy, P. Balaprakash, D. Livescu: Time-series learning of latent-space dynamics for reduced-order model closure, {\it Physica D: Nonlinear Phenomena}, 405, 132368, 2020. 

\item Y. Hossain, \textbf{R. Maulik}, H. Park, M. Ahmed, C. Bach, O. San: Improvement of Unitary Equipment and Heat Exchanger Testing Methods, {\it ASHRAE Transactions}, 125.2, 2019. 

\item \textbf{R. Maulik}, O. San, J. Jacob, C. Crick: Online turbulence model classification for large eddy simulation using deep learning, {\it Journal of Fluid Mechanics}, 870, 784-812, 2019. 

\item O. San, \textbf{R. Maulik}, M. Ahmed: An artificial neural network framework for reduced order modeling of transient flows, {\it Communications in Nonlinear Science and Numerical Simulation}, 77, 271-287, 2019. 

\item \textbf{R. Maulik}, O. San, A. Rasheed, P. Vedula: Subgrid modeling for two-dimensional turbulence using artificial neural networks, {\it Journal of Fluid Mechanics}, 858, 122-144, 2019. 

\item \textbf{R. Maulik}, O. San, A. Rasheed, P. Vedula: Data-driven deconvolution for large eddy simulation of Kraichnan turbulence, {\it Physics of Fluids}, 30, 125109, 2018. 

\item O. San, \textbf{R. Maulik}: Stratified Kelvin-Helmholtz turbulence of compressible shear flows, {\it Nonlinear Processes in Geophysics}, 25, 457--476, 2018.

\item O. San, \textbf{R. Maulik}: Extreme learning machine for reduced order modeling of turbulent geophysical flows, {\it Physical Review E}, 97, 042322, 2018. 

\item O. San, \textbf{R. Maulik}: Machine learning closures for model order reduction of thermal fluids,  {\it Applied Mathematical Modelling}, 60, 681-710, 2018. 

\item \textbf{R. Maulik}, O. San, R. Behera : An adaptive multilevel wavelet framework for scale-selective WENO reconstruction schemes, {\it International Journal of Numerical Methods in Fluids}, 87 (5), 239-269, 2018. 

\item O. San, \textbf{R. Maulik}: Neural network closure models for nonlinear model order reduction, {\it Advances in Computational Mathematics}, 44, 1717-1750, 2018. 

\item \textbf{R. Maulik}, O. San: A dynamic closure modeling framework for large eddy simulation using approximate deconvolution: Burgers equation, {\it Cogent Physics}, 5, 1464368, 2018. 

\item \textbf{R. Maulik}, O. San: A neural network approach for the blind deconvolution of turbulent flows, {\it Journal of Fluid Mechanics}, 831, 151-181, 2017. 

\item \textbf{R. Maulik}, O. San: A novel dynamic framework for subgrid-scale parametrization of mesoscale eddies in quasigeostrophic turbulent flows, {\it Computers and Mathematics with Applications}, 74, 420-445, 2017. 

\item \textbf{R. Maulik}, O. San: Explicit and implicit LES closures for Burgers turbulence, {\it Journal of Computational and Applied Mathematics}, 327, 12-40, 2017. 

\item \textbf{R. Maulik}, O. San: Resolution and energy dissipation characteristics of implicit LES and explicit filtering models for compressible turbulence, {\it Fluids}, 2(2)-14, 2017. 

\item \textbf{R. Maulik}, O. San: A dynamic subgrid-scale modeling framework for Boussinesq turbulence, {\it International Journal of Heat and Mass Transfer}, 108, 1656-1675, 2017. 

\item \textbf{R. Maulik}, O. San: A dynamic framework for scale-aware parameterizations of eddy viscosity coefficient in two-dimensional turbulence, {\it International Journal of Computational Fluid Dynamics}, 31(2), 69-92, 2017. 

\item \textbf{R. Maulik}, O. San: A stable and scale-aware dynamic modeling framework for subgrid-scale parameterizations of two-dimensional turbulence, {\it Computers \& Fluids} 158, 11-38, 2016. 

\item \textbf{R. Maulik}, O. San: Dynamic modeling of the horizontal eddy viscosity coefficient for quasigeostrophic ocean circulation problems, {\it Journal of Ocean Engineering and Science} 1, 300-324, 2016.

\item H. H. Marbini, \textbf{R. Maulik}: A biphasic transversely isotropic poroviscoelastic model for the unconfined compression of hydrated soft tissue, {\it Journal of Biomechanical Engineering} 138, 031003, 2016.

\end{enumerate}

\subsection*{Peer-reviewed conference publications}

\begin{enumerate}

\item Sudeepta Mondal, Gina M. Magnotti, Bethany Lusch, \textbf{Romit Maulik}, Roberto Torelli: Machine Learning-Enabled Prediction of Transient Injection Map in Automotive Injectors With Uncertainty Quantification, ASME Internal Combustion Engine Fall Conference, 2021. (Accepted)

\item V. Sastry, \textbf{R. Maulik}, V. Rao, B. Lusch, S. Renganathan, R. Kotamarthi: Data-driven deep learning emulators for geophysical forecasting, {\it International Conference on Computational Science, 433-446, 2021}, \\ \texttt{https://doi.org/10.1007/978-3-030-77977-1\_35}. Acceptance rate: 30.7\%.

\item \textbf{R. Maulik}, R. Egele, B. Lusch,  P. Balaprakash: Recurrent neural network architecture search for geophysical emulation, {\it Proceedings of the International Conference for High Performance Computing, Networking, Storage and Analysis (SC), 2020, 10.5555/3433701.3433711}. Acceptance rate: ~20\%.

\item V. Rao, \textbf{R. Maulik}, E. Constantinescu, M. Anitescu: A machine learning method for computing rare event probabilities, {\it International Conference on Computational Science, 169-182, 2020}, \\ \texttt{https://link.springer.com/chapter/10.1007\%2F978-3-030-50433-5\_14}. Acceptance rate: 30.7\%.

\item \textbf{R. Maulik}, O. San, C. Bach: A computational investigation of the effect of ground clearance in vertical ducting systems, 2018, Purdue University, Herrick Labs Conferences 2018. \\ \texttt{https://docs.lib.purdue.edu/ihpbc/308/}.

\end{enumerate}

\subsection*{Conference publications}

\begin{enumerate}

\item \textbf{R. Maulik}, H. Sharma, S. Patel, B. Lusch, E. Jennings: Deploying deep learning in OpenFOAM with TensorFlow: A tutorial, AIAA SciTech Forum 2021, \texttt{https://doi.org/10.2514/6.2021-1485}.

\item P. Milan, R. Torelli, B. Lusch, \textbf{R. Maulik}, G. Magnotti: Data-Driven Modeling of Large-Eddy Simulations for Fuel Injector Design, AIAA SciTech Forum 2021, \\ \texttt{https://doi.org/10.2514/6.2021-1016}.

\item \textbf{R. Maulik}, V. Rao, S. Renganathan, S. Letizia, G. Iungo: Cluster analysis of wind turbine wakes measured through a scanning Doppler wind LiDAR, AIAA SciTech Forum 2021, \\ \texttt{https://doi.org/10.2514/6.2021-1181}.

\end{enumerate}

\subsection*{Peer-reviewed workshop proceedings}

\begin{enumerate}

\item D. Skinner, \textbf{R. Maulik}: Meta-modeling strategy for data-driven forecasting, \textit{Tackling Climate Change with Machine Learning Workshop, NeurIPS}, 2020. \url{https://www.climatechange.ai/papers/neurips2020/13.html}. 

\item N. Garland, \textbf{R. Maulik}, Q. Tang, X. Tang, P. Balaprakash: Progress towards high fidelity collisional-radiative model surrogates for rapid in-situ evaluation, \textit{Machine Learning for Physical Sciences Workshop, NeurIPS}, 2020. \url{https://ml4physicalsciences.github.io/2020/files/NeurIPS_ML4PS_2020_79.pdf}.

\item K. Fukami, \textbf{R. Maulik}, N. Ramachandra, K. Fukagata, K. Taira: Probabilistic neural network-based reduced-order surrogate for fluid flows, \textit{Machine Learning for Physical Sciences Workshop, NeurIPS}, 2020. \url{https://ml4physicalsciences.github.io/2020/files/NeurIPS_ML4PS_2020_7.pdf}

\item \textbf{R. Maulik}, R. S. Assary, P. Balaprakash: Site-specific graph neural network for predicting protonation energy of oxygenate molecules, {\it Machine Learning for Physical Sciences Workshop, NeurIPS}, 2019. \url{https://ml4physicalsciences.github.io/2019/files/NeurIPS_ML4PS_2019_134.pdf}

\item \textbf{R. Maulik}, V.Rao, S. Madireddy, B. Lusch, P. Balaprakash: Using recurrent neural networks for nonlinear component computation in advection-dominated reduced-order models, \textit{Machine Learning for Physical Sciences Workshop, NeurIPS}, 2019. \url{https://ml4physicalsciences.github.io/2019/files/NeurIPS_ML4PS_2019_99.pdf}.

\end{enumerate}



\section*{Talks presented}

\begin{enumerate}


\item Parallelized emulator discovery and uncertainty quantification using DeepHyper, Mechanistic Machine Learning and Digital Twins for Computational Science, Engineering \& Technology, September 26-29, 2021.

\item Modified neural ordinary differential equations for stable learning of chaotic dynamics, \textbf{Invited talk}, Applied Mathematics Seminar Series, Texas Tech University, September 1, 2021.

\item Neural architecture search for surrogate modeling, \textbf{Invited talk}, ML4I Forum, Lawrence Livermore National Laboratory, August 10-12, 2021.

\item Scalable scientific machine learning for computational fluid dynamics, \textbf{Plenary talk}, Computational Sciences and AI in Industry, June 7-9, 2021.

\item Neural architecture search for surrogate modeling, \textbf{Invited talk}, DDPS Seminar Series, Lawrence Livermore National Laboratory, May 27, 2021.

\item Incorporating inductive biases for the surrogate modeling of dynamical systems, \textbf{Invited talk} at Machine Learning for Dynamical Systems Special Interest Group, Alan Turing Institute, Imperial College London, April 14, 2021.

\item Surrogate modeling with learned kernels (Kernel Flows), \textbf{Invited talk}, Uncertainty Quantification in Climate Science, NASA JPL Climate Center Virtual Workshop, March 24, 2021.

\item Scalable recurrent neural architecture search for geophysical emulation, \textbf{Invited talk} at SIAM-CSE Minisymposium on Physics-Guided Machine Learning and Data-Driven Methods in Computational Geoscience.

\item Incorporating Inductive Biases as Hard Constraints for Scientific Machine Learning, MCS-LANS seminar, Argonne National Laboratory, February 2021.

\item Scalable scientific machine learning for computational fluid dynamics, \textbf{Invited talk}, Department of Mechanical Engineering, The City College of New York, October, 2020.

\item Data-driven model order reduction for geophysical emulation. \textbf{Invited talk} at the Second Symposium on Machine Learning and Dynamical Systems, Fields Institute, Toronto, September, 2020.

\item Scalable scientific machine learning for computational fluid dynamics, \textbf{Invited talk}, Department of Mechanical Engineering, Rice University, September, 2020.

\item Machine Learning Enablers for System Optimization and Design, MCS-LANS seminar, Argonne National Laboratory, August 2020.

\item Surrogate-based machine-learning for system optimization and design, \textbf{Invited talk} at Los Alamos National Laboratory for Tokamak Disruption Simulation (TDS) working group, August, 2020.

\item Non-intrusive reduced-order model search for geophysical emulation, \textbf{Guest lecture}, MAE259a: Data science for fluid dynamics (offered by Kunihiko Taira), University of California Los Angeles, June 2020.

\item Spatiotemporally dynamic implicit large eddy simulation using machine learning classifiers, Session on Domain-Aware, Interpretable and Robust Scientific Machine Learning Methods Applied to Computational Mechanics, AIAA Aviation Forum, June, 2020, Reno. 

\item Machine learning for computational fluid dynamics, \textbf{Invited talk} at PyData Meetup Chicago, May 2020.

\item Recurrent neural architecture search for geophysical emulation using DeepHyper, \textbf{Invited talk} at AI-HPC seminar, Argonne National Laboratory, April, 2020.

\item Machine Learned Reduced-Order Models for Advective Partial Differential Equations, MCS-LANS seminar, Argonne National Laboratory, February, 2020.

\item Machine Learned Reduced-Order Models for Advective Partial Differential Equations, 2020 Spring Multiscale Seminar, Illinois Institute of Technology, Chicago, February, 2020.

\item General purpose data science for general purpose CFD: Integrating Tensorflow into OpenFOAM at scale, Workshop for Machine Learning for Transport Phenomena, February, 2020, Dallas.

\item Machine learning of sequential data for non-intrusive reduced-order models, Bulletin of the American Physical Society 72, November, 2019.

\item Tackling the limitations of conventional ROMs for advection-dominated nonlinear dynamical systems using machine learning, \textbf{Invited talk}, Advanced Statistics meets Machine Learning-III workshop, Argonne National Laboratory, November, 2019.

\item Data-driven sub-grid models for the large-eddy simulation of turbulence, John Zink Hamworthy Combustion, Tulsa, August, 2019.

\item Novel turbulence closures using physics-informed machine learning, Argonne Physical Sciences and Engineering Division AI Townhall, July, 2019. 

\item Data-driven deconvolution for the sub-grid modeling of large eddy simulations of two-dimensional turbulence, SIAM-CSE, March, 2019.

\item Data-driven deconvolution for the large eddy simulation of Kraichnan turbulence, Bulletin of the American Physical Society 71, November, 2018.

\item A computational investigation of the effect of ground clearance in vertical ducting systems, 2018, Purdue University, Herrick Labs Conferences, July, 2018. 

\item A neural network approach for the blind deconvolution of turbulent flows, Bulletin of the American Physical Society 70, November, 2017.

\item A generalized wavelet based grid-adaptive and scale-selective implementation of WENO schemes for conservation laws, Texas Applied Mathematics and Engineering Symposium, The University of Texas, Austin, September 2017.

\item An explicit filtering framework based on Perona-Malik anisotropic diffusion for shock capturing and subgrid scale modeling of Burgers' turbulence, Bulletin of the American Physical Society 69, November, 2016.

\item A dynamic hybrid subgrid-scale modeling framework for large eddy simulations, Bulletin of the American Physical Society 69, November, 2016.

\end{enumerate}

% \section*{Other participation in workshops}

% \begin{itemize}
%     \item Invited participant, Vistas in the Applied Mathematical Sciences, Institute for Mathematical Statistical Innovation (IMSI),  The University of Chicago, IL, 2020.

%     \item Invited participant, NSF workshop on Machine Learning for Transport Phenomena, Southern Methodist University, TX, 2020.

%     \item Invited participant, Mathematics of Reduced Order Models, ICERM, Brown University, RI, 2020.

%     \item Invited participant, Algorithms for Dimension and Complexity Reduction, ICERM, Brown University, RI, 2020.

%     \item Invited participant, IPAM Workshop III: Validation and Guarantees in Learning Physical Models: from Patterns to Governing Equations to Laws of Nature, UC Los Angeles, October 2019.
    
%     \item Invited participant, Department of Energy - AI for Science Townhall, Argonne National Laboratory, June 2019.
    
%     \item Invited participant, Advances in PDEs: Theory, Computation and Application to CFD, ICERM, Brown University, RI, 2018.
    
%     \item Invited participant, SDSC Summer program in HPC and Data Science, UC San Diego, 2017.
% \end{itemize}

\section*{Professional service}

\subsection*{Tutorials organized}

\begin{itemize}
  \item Tutorial lead - Autoencoders for PDE surrogate models, ATPESC 2020.
  \item Tutorial lead - Statistical methods for machine learning, ALCF AI4Science tutorial 2019, Argonne National Laboratory.
  \item Tutorial lead - DeepHyper for scalable hyperparameter and neural architecture search on ALCF machines, ALCF Simulation Data and Learning workshop 2019, Argonne National Laboratory.
  \item TensorFlow workshop, Mechanical \& Aerospace Engineering, Oklahoma State University, 2018.
  \item An introduction to high performance computing for middle school kids, National Lab Day, Oklahoma State University 2017, 2018.
\end{itemize}

\subsection*{Minisymposia}

\begin{itemize}
  \item Organizer - Acceleration and Enhancement of High-fidelity PDE Solvers through Machine Learning, 16th U.S. National Congress on Computational Mechanics, IL, 2021.
  \item Co-organizer - Argonne National Laboratory - AI, Statistics and Machine Learning Journal Club.
  \item Session chair - Domain-Aware, Interpretable and Robust Scientific Machine Learning Methods Applied to Computational Mechanics, AIAA Aviation Forum, Reno, NV, 2020.
  \item MS Organizer \& Session chair - Domain-Aware, Interpretable and Robust Machine Learning for Computational Science, SIAM-CSE, 2021.
  \item MS Organizer \& Session chair - Machine Learning methods in Computational Fluid Dynamics, SIAM-CSE, Spokane, WA, 2019.
  \item Session chair - MAE Graduate Research Symposium, Oklahoma State University, 2018.
\end{itemize}

\subsection*{Journal Review}

\begin{itemize}
  \item Journal reviewer for - AIAA Journal, Applied Mathematical Modeling, Chaos, Computer Methods in Applied Mathematics and Engineering, Communications in Computational Physics, Computers and Fluids, Computer Physics Communications, International Journal of Computational Fluid Dynamics, IEEE Transactions on Plasma Science, Journal of Fluid Mechanics, Journal of Scientific Computing, Physics of Fluids, Physica D, International Journal of Numerical Methods in Fluids, Journal of Computational Physics, Journal of Nonlinear Science, Nature Communications, Nature Machine Intelligence, Nature Scientific Reports, Theoretical and Computational Fluid Dynamics, Atmospheric Science Letters, New Journal of Physics.
\end{itemize}

\section*{Software developed}

\begin{enumerate}

\item R. Maulik, H. Sharma, S. Patel, B. Lusch, E. Jennings, TensorFlowFoam: A framework that enables the deployment of deep learning (in Python) and partial differential equation solutions concurrently in OpenFOAM - a C++-based open-source finite-volume based computational physics package. \url{https://github.com/argonne-lcf/TensorFlowFoam}.

\item R. Maulik, S. Pawar, PAR-RL: A framework that leverages the Ray library to deploy scalable deep reinforcement learning for arbitrary scientific environments on leadership class machines. Tested on ALCF supercomputer Theta for controlling simulations of dynamical systems. \url{https://github.com/Romit-Maulik/PAR-RL}.

\item R. Maulik, G. Mengaldo, PyParSVD: A Parallelized, streaming, and randomized implementation of the SVD for Python using mpi4py. \url{https://github.com/Romit-Maulik/PyParSVD}. DOI: 10.5281/zenodo.4562889.

\end{enumerate}

\section*{Funding \& support}

\subsection*{Funded}

\begin{itemize}

  \item (Active) Inertial neural surrogates for stable dynamical prediction, U.S. Department of Energy (Advanced Scientific Computing Research), Role: PI; Year 2021-2024. Amount: \$3457000.

  \item (Active) A Scalable, Energy Efficient HPC Environment for AI-Enabled Science, National Science Foundation Collaborative PPoSS funding. Role: Co-PI; Year 2021-2022. Amount: \$150,000.

  \item (Active) SambaWF: Highly resolved surrogate models for weather forecasting, LDRD-Expedition, Argonne National Laboratory, U.S. Department of Energy. Role: PI; Year: 2021-2021. Amount: \$50,000. 

  \item (Active) AI emulator assisted data assimilation, Future computing, LDRD-Prime, Argonne National Laboratory, U.S. Department of Energy. Role: Co-PI; Year: 2021--2023. Amount: \$1240000. 

  \item (Active) RAPIDS2: A SciDAC Institute for Computer Science, Data, and Artificial Intelligence, U.S. Department of Energy. Role: Senior Personnel; Year: 2020--2025. Amount: \$4236400.

  \item (Finished) RAPIDS: A SciDAC Institute for Computer Science and Data, and Artificial Intelligence, U.S. Department of Energy. Role: Senior Personnel; Year 2019-2020.


\end{itemize}

% \subsection*{Unfunded}

% \begin{itemize}

%   \item Improve Earth System model predictability by interpretable artificial intelligence for wildfire and compound extreme prediction and the impact on electricity grid, Climate and Energy Action, LDRD-Prime, Argonne  National Laboratory, U.S. Department of Energy. Role: Co-PI.

%   \item Closure Model Development with deep Reinforcement Learning and High-fidelity, High-void-fraction Experiment Data, U.S. Department of Energy Nuclear Energy University Program, Nuclear Energy Advanced Modeling and Simulation-4. Role: Co-PI, 2020.

%   \item DeepRefinement: Machine-Learned Adaptive Mesh Refinement for Predictive Scientific Computing, U.S. Department of Energy ASCR LAB 20-2319, Role: PI, 2020.

%   \item Machine learning linear solvers for accelerating numerical methods, LDRD-Seed. Role: PI, 2019.

%   \item Forecasting Evolutionary Distributions Using Generative Machine Learning Models, NSF-DMS. Role: Co-PI, 2021.

% \end{itemize}

\bigskip

% Footer
%\begin{center}
%  \begin{footnotesize}
%    Last updated: \today \\
%    \href{\footerlink}{\texttt{\footerlink}}
%  \end{footnotesize}
%\end{center}

\end{document}
