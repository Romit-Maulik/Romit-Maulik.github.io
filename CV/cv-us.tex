% LaTeX Curriculum Vitae Template
%
% Copyright (C) 2004-2009 Jason Blevins <jrblevin@sdf.lonestar.org>
% http://jblevins.org/projects/cv-template/
%
% You may use use this document as a template to create your own CV
% and you may redistribute the source code freely. No attribution is
% required in any resulting documents. I do ask that you please leave
% this notice and the above URL in the source code if you choose to
% redistribute this file.

\documentclass[letterpaper]{article}

\usepackage{hyperref}
\usepackage{geometry}

% Comment the following lines to use the default Computer Modern font
% instead of the Palatino font provided by the mathpazo package.
% Remove the 'osf' bit if you don't like the old style figures.
\usepackage[T1]{fontenc}
\usepackage[sc,osf]{mathpazo}

% Set your name here
\def\name{Romit Maulik}

% Replace this with a link to your CV if you like, or set it empty
% (as in \def\footerlink{}) to remove the link in the footer:
%\def\footerlink{http://jblevins.org/projects/cv-template/}

% The following metadata will show up in the PDF properties
\hypersetup{
  colorlinks = true,
  urlcolor = black,
  pdfauthor = {\name},
  pdfkeywords = {},
  pdftitle = {\name: Curriculum Vitae},
  pdfsubject = {Curriculum Vitae},
  pdfpagemode = UseNone
}

\geometry{
  body={6.5in, 8.5in},
  left=1.0in,
  top=1.25in
}

% Customize page headers
\pagestyle{myheadings}
\markright{\name}
\thispagestyle{empty}

% Custom section fonts
\usepackage{sectsty}
\sectionfont{\rmfamily\mdseries\Large}
\subsectionfont{\rmfamily\mdseries\itshape\large}

% Other possible font commands include:
% \ttfamily for teletype,
% \sffamily for sans serif,
% \bfseries for bold,
% \scshape for small caps,
% \normalsize, \large, \Large, \LARGE sizes.

% Don't indent paragraphs.
\setlength\parindent{0em}

% Make lists without bullets
\renewenvironment{itemize}{
  \begin{list}{}{
    \setlength{\leftmargin}{1.5em}
  }
}{
  \end{list}
}

\begin{document}

% Place name at left
{\huge \name}

% Alternatively, print name centered and bold:
%\centerline{\huge \bf \name}

\vspace{0.25in}

\begin{minipage}{0.45\linewidth}
  Information Sciences and Technology, \\
  E327, Westgate Building, \\
  University Park, PA 16802.
\end{minipage}
\begin{minipage}{0.45\linewidth}
  \begin{tabular}{ll}
    Phone: & (405) 982-0161 \\
    Email: & \href{mailto:rmaulik@psu.edu}{\tt rmaulik@psu.edu} \\
    Homepage: & \href{https://romit-maulik.github.io/}{\tt romit-maulik.github.io} \\
    DOB: & 11-17-1989
  \end{tabular}
\end{minipage}

\section*{Research interests}

Scientific machine learning,  high-performance computing, computational fluid dynamics, reduced-order modeling, numerical methods, stochastic processes.

\section*{Positions held}
\begin{itemize}

\item  \textbf{July 2023 - Present} \\ Assistant Professor \& Institute of Computational and Data Sciences Co-Hire, Information Sciences and Technology, Pennsylvania State University \\ \& \\ Joint Appointment, Mathematics and Computer Science Division, Argonne National Laboratory.

\item  \textbf{Jun, 2021 - Jun 2023} \\ Assistant Computational Scientist, Mathematics and Computer Science, Argonne National Laboratory.

\item \textbf{Oct, 2020 - Jun 2023} \\ Research Assistant Professor, Department of Applied Mathematics, IIT-Chicago.

\item \textbf{Jun, 2019 - May, 2021} \\ Margaret Butler Postdoctoral Fellow, Leadership Computing Facility, Argonne National Laboratory.

\item \textbf{Jan, 2019 - May, 2019} \\ Predoctoral Appointee - Mathematics and Computer Science Division, Argonne National Laboratory.  

\item \textbf{Jan, 2016 - Jan, 2019} \\ Research Assistant - Computational Fluid Dynamics Laboratory, Oklahoma State University. 

\item \textbf{Aug, 2013 - Jul, 2015} \\ Research Assistant - Computational Biomechanics Laboratory, Oklahoma State University. 

\item  \textbf{Jan, 2013 - Dec, 2018} \\ Teaching Assistant (Introductory Dynamics, Introductory Fluid Mechanics, Practical CFD) - Mechanical \& Aerospace Engineering, Oklahoma State University.

\item \textbf{Aug, 2012 - Aug, 2013} \\ Design Engineer - Tata Technologies Limited, India. 
\end{itemize}

\section*{Education}

\begin{itemize}
  \item PhD. Mechanical \& Aerospace Engineering, Oklahoma State University. \hfill \textbf{2016-2019}
  \item M.S. Mechanical \& Aerospace Engineering, Oklahoma State University. \hfill \textbf{2013-2015}
  \item B.E. Mechanical Engineering, Birla Institute of Technology, India. \hfill \textbf{2008-2012}
\end{itemize}


\section*{Honors \& awards}

\begin{itemize}

\item Best paper award, CASML IISC Bangalore, 2024, ``Improved deep learning of chaotic dynamical systems with multistep penalty losses".

\item Early Career Award, Army Research Office, 2024-2027.

\item Best paper award, May 2024, ICLR 2024 Workshop on Climate Change and AI: `Scaling transformer neural networks for skillful and reliable medium-range weather forecasting'.

\item Impact Argonne Award, September 2023: `For developing an AI model for climate and creating an innovated data assimilation methodology'.

\item Best paper award - Machine Learning-Enabled Prediction of Transient Injection Map In Automotive Injectors With Uncertainty Quantification, 2021 ASME Internal Combustion Engine Fall Conference, \url{https://doi.org/10.1115/ICEF2021-67888}.

\item Impact Argonne Award, September, 2021: `For tackling several climate model challenges and advancing the field of downscaled climate modeling and impact analysis'.

\item Editor's pick - Reduced-order modeling of advection-dominated systems with recurrent neural networks and convolutional autoencoders , {\it Physics of Fluids}, \url{https://doi.org/10.1063/5.0039986}.

\item Editor's pick - Non-autoregressive time-series methods for stable parametric reduced-order models , {\it Physics of Fluids}, \url{https://doi.org/10.1063/5.0019884}.

\item \href{https://www.alcf.anl.gov/margaret-butler-fellowship-computational-science}{Margaret Butler Fellow}, Argonne Leadership Computing Facility, Argonne National Laboratory, 2019-2021.

\item SIAM Travel Grant: 2019 SIAM Conference on Computational Science and Engineering, Spokane, WA, 2019.

\item Outstanding Graduate Student, College of Engineering Architecture and Technology, Oklahoma State University, 2018.

\item Graduate College Robberson Summer Research Fellowship, Oklahoma State University, 2018.

\item SIAM TX-LA Section Travel Grant, Texas Applied Mathematics and Engineering Symposium, 2017.

\item Graduate Student Travel Grant, American Physical Society - Division of Fluid Dynamics, 2017.

\item Graduate Student Travel Grant, Graduate Program Student Government Authority, Oklahoma State University, 2017.

\item FGSA Travel Grant for Excellence in Graduate Research, American Physical Society, 2017.

\item John Brammer Fellowship, Oklahoma State University, 2016.

\item Graduate College Top Tier Fellowship, Oklahoma State University, 2016.

\end{itemize}

\section*{Funding \& support}

\begin{itemize}

  \item (Active, 2025-2027) Machine learning to improve cycling and forecasts with GEOS and expedite the evaluation of assimilating observations from new instruments, AIST Earth System Digital Twin Award, NASA, Role: PI.

  \item (Active, 2024-2026) Black hole tomography with gravitational waves and artificial intelligence, New Initiative Grant, Charles E. Kaufman Foundation, Role: Co-PI.

  \item (Active, 2024-2027) EARLY CAREER PROGRAM: RIFLES: Rapid Information Fusion through Learning Effective ClosureS, Army Research Office (Modeling of Complex Systems), Role: PI.

  \item (Active, 2024-2027) Using Machine Learning to Uncover Deep Convective Cloud Processes with TRACER Observations, U.S. Department of Energy (Atmospheric System Research), Role: Co-PI.

  \item (Active, 2021-2025) Inertial neural surrogates for stable dynamical prediction, U.S. Department of Energy (Advanced Scientific Computing Research), Role: Institutional PI.

  \item (Active, 2023-2026) DeepFusion Accelerator for Fusion Energy Sciences in Disruption Mitigations, U.S. Department of Energy (Fusion Energy Sciences), Role: Institutional PI.

  \item (Active, 2020-2025) RAPIDS2: A SciDAC Institute for Computer Science, Data, and Artificial Intelligence, U.S. Department of Energy (Advanced Scientific Computing Research). Role: Institutional PI.

  \item (Active, 2024-2026) Koopman-operator learning for stable and physical dynamical systems forecasting, Los Alamos National Laboratory Subaward. Role: Institutional PI.

  \item (Active, 2024-2025) Foundation models for surface hydrology, Argonne National Laboratory Subaward. Role: Institutional PI.

  \item (Finished, 2021-2023) AI emulator assisted data assimilation, Future computing, LDRD-Prime, Argonne National Laboratory, U.S. Department of Energy. Role: Co-PI (Argonne National Laboratory).

  \item (Finished, 2022-2023) A Scalable, Energy Efficient HPC Environment for AI-Enabled Science, National Science Foundation Collaborative PPoSS funding. Role: Co-PI (IIT-Chicago).

  \item (Finished, 2021) SambaWF: Highly resolved surrogate models for weather forecasting, LDRD-Expedition, Argonne National Laboratory, U.S. Department of Energy. Role: PI (Argonne National Laboratory).

  \item (Finished, 2019-2021) Scalable machine learning for turbulence closure and reduced-order modeling, Margaret Butler Fellowship, Role: PI (Argonne National Laboratory).


\end{itemize}


\section*{Supervision}

Asterisks indicate active supervision.

\begin{itemize}

  \item Dr. Shivam Barwey$^\ast$, AETS Named Postdoctoral Fellow, Argonne National Laboratory, 2022-Present.

  \item Dr. James Henry$^\ast$, Postdoctoral Scholar, Pennsylvania State University, 2024-Present.

  \item Dr. Xuyang Li$^\ast$, Postdoctoral Scholar, Pennsylvania State University, 2024-Present.

  \item Dibyajyoti Chakraborty$^\ast$, PhD Student, Pennsylvania State University, 2023-Present.

  \item Haiwen Guan$^\ast$, PhD Student, Pennsylvania State University, 2024-Present.

  \item Hojin Kim$^\ast$, PhD Student, Pennsylvania State University, 2024-Present.

  \item Matthew Poska$^\ast$, Graduate Student, Pennsylvania State Univeristy \& Argonne National Laboratory. DOE SCGSR Fellow, 2022-Present (co-advised with Dr. Sharon Huang, IST, Penn State).

  \item Zachariah Malik, Visiting graduate Student, Argonne National Laboratory, NSF MSGI Fellow, Summer 2023, 2024.

  \item Vinamr Jain, Undergraduate Intern, IIT-Delhi, 2023-2024.

  \item Aaryan Bavishi, Undergraduate Student, Pennsylvania State University, 2023-Present.

  \item Abhinab Bhattacharjee, Graduate Student, Argonne National Laboratory, Givens Associate, Summer 2023.

  \item Deepinder Jot Singh Aulakh, Graduate Student, Argonne National Laboratory, ALCF Research Aide, Summer 2023.

  \item Trent Gerew, Undergraduate Student, Argonne National Laboratory, DOE SULI Program, Spring 2023-Present.

  \item Jonah Botvinick-Greenhouse, Visiting Graduate Student, Cornell University \& Argonne National Laboratory, NSF MSGI Fellow, Summer 2022.

  \item Sen Lin, Graduate Student, Argonne National Laboratory, Givens Associate, Summer 2022.

  \item Cyril Le Doux, Undergraduate Student, Argonne National Laboratory, DOE SULI Program, Summer 2022.

  \item Gurpreet Singh Hora, Graduate Student, Columbia University, With Laurent White at AMD Research, Summer 2022.

  \item Sahil Bhola, Graduate Student, Argonne National Laboratory, ALCF Research Aide, Summer 2021.

  \item Alec Linot, Graduate Student, Argonne National Laboratory, Givens Associate, Summer 2021.

  \item William McClure, Graduate Student, IIT-Chicago, Masters Thesis, 2021. 

  \item Janah Richardson, High-school student, Afro-Academic, Cultural, Technological and Scientific Olympics (ACT-SO) High School Research Program, 2020-2021. Gold medal winner in Computer Science category- Illinois.

  \item Suraj Pawar, Graduate Student, Argonne National Laboratory, ALCF Research Aide, Summer 2020.

  \item Dominic Skinner, Graduate Student, Argonne National Laboratory, NSF MSGI Fellow, Summer 2020.

\end{itemize}


\section*{Publications}

\subsection*{Under review}

\begin{enumerate}

\item X. Li, \textbf{R. Maulik}: SALSA-RL: Stability Analysis in the Latent Space of Actions for Reinforcement Learning, {\it 2502.15512}.

\item X. Wang, J. Yang, J. Adie, K. Furtado, C. Chen, T. Arcomano, \textbf{R. Maulik}, G. Mengaldo: CondensNet: Enabling stable long-term climate simulations via hybrid deep learning models with adaptive physical constraints, {\it arXiv:2502.13185}.

\item H. Guan, T. Arcomano, A. Chattopadhyay, \textbf{R. Maulik}: LUCIE: A Lightweight Uncoupled ClImate Emulator with long-term stability and physical consistency for O(1000)-member ensembles, {\it arXiv:2405.16297}

\item J. Botvinick-Greenhouse, M. Oprea, \textbf{R. Maulik}, Y. Yang: Measure-Theoretic Time-Delay Embedding: Analysis and Application, {\it arXiv:2409.08768}.

\item Z. Malik, \textbf{R. Maulik}: A competitive baseline for deep learning enhanced data assimilation using conditional Gaussian ensemble Kalman filtering, {\it arXiv:2409.14300}.

\item S. Barwey, P. Pal, S. Patel, R. Balin, B. Lusch, V. Vishwanath, \textbf{R. Maulik}, R. Balakrishnan: Mesh-based Super-Resolution of Fluid Flows with Multiscale Graph Neural Networks, {\it arXiv:2409.07769}.

\item V. Jain, \textbf{R. Maulik}: Higher order quantum reservoir computing for non-intrusive reduced-order models, {\it arXiv:2407.21602}.

\item D. Chakraborty, S. Barwey, H. Zhang, \textbf{R. Maulik}: A note on the error analysis of data-driven closure models for large eddy simulations of turbulence, {\it arXiv:2405.17612}.

\item A. Nair, S. Barwey, P. Pal, J. MacArt, \textbf{R. Maulik}: Understanding Latent Timescales in Neural Ordinary Differential Equation Models for Advection-Dominated Dynamical Systems, {\it arXiv:2403.02224}.

\item H. Kim, V. Shankar, V. Vishwanathan, \textbf{R. Maulik}: Generalizable data-driven turbulence closure modeling on unstructured grids with differentiable physics, {\it arXiv:2307.13533}.

\item C. Moss, \textbf{R. Maulik}, G. V. Iungo: Modeling Wind Turbine Performance and Wake Interactions with Machine Learning, {\it arXiv:2212.01483}.

\item V. Shankar, S. Barwey,  Z. Kolter, \textbf{R. Maulik}, V. Viswanathan: Importance of equivariant and invariant symmetries for fluid flow modeling, , {\it arxiv:2307.05486}.

\end{enumerate}


\subsection*{Peer-reviewed journal articles}

\begin{enumerate}

\item V. Shankar, D. Chakraborty, V. Vishwanathan, \textbf{R. Maulik}: Differentiable turbulence: Closure as PDE-constrained optimization, {\it Physical Review Fluids, Accepted}.

\item T. Chang, A. Gillette, \textbf{R. Maulik}: Leveraging Interpolation Models and Error Bounds for Verifiable Scientific Machine Learning, {\it Journal of Computational Physics}, Volume 524, 1 March 2025, 113726.

\item S. Barwey, H. Kim, \textbf{R. Maulik}: Interpretable A-posteriori Error Indication for Graph Neural Network Surrogate Models, {\it Computer Methods in Applied Mechanics and Engineering}, Volume 433, Part B, 1 January 2025, 117509.

\item D. Chakraborty, S. Chung, \textbf{R. Maulik}: Divide and conquer: Learning chaotic dynamical systems with multistep penalty neural ordinary differential equations, {\it Computer Methods in Applied Mechanics and Engineering}, Volume 432, Part A, 1 December 2024, 117442.

\item B. Sanderse, P. Stinis, \textbf{R. Maulik}, S. Ahmed: Scientific machine learning for closure models in multiscale problems: a review, {\it Foundations of Data Science}, 2024.

\item C. Moss, \textbf{R. Maulik}, V. Iungo: Augmenting Insights from Wind Turbine Data through Data-Driven Approaches, {\it Applied Energy}, 376, 124116, 2024.

\item D. Aulakh, X. Yang, \textbf{R. Maulik}: Robust experimental data assimilation for the Spalart-Allmaras turbulence model, {\it Physical Review Fluids}, 9, 084608, 2024.

\item S. Yang, H. Kim, Y. Hong, K. Yee, \textbf{R. Maulik}, N. Kang:  Data-driven Physics-Informed Neural Networks: A Digital Twin Perspective, {\it Computer Methods in Applied Mechanics and Engineering}, Volume 428, 117075, 2024.

\item M. Rogowski, B. Yeung, O. Schmidt, \textbf{R. Maulik}, M. Parsani, L. Dalcin, G. Mengaldo: Unlocking massively parallel spectral proper orthogonal decompositions in the PySPOD package, {\it Computer Physics Communications}, Volume 302, 109246, 2024.

\item K. Asztalos, R. Steijl, \textbf{R. Maulik}: Reduced-order modeling on a near-term quantum computer, {\it Journal of Computational Physics}, Volume 510, 113070, 2024.

\item C. Moss, \textbf{R. Maulik}, G. Iungo: A Call for Enhanced Data-Driven Insights into Wind Energy Flow Physics, {\it Theoretical and Applied Mechanics Letters}, Volume 14, Issue 1, 2024, 100488.

\item S. Barwey, V. Shankar, V. Vishwanathan, \textbf{R. Maulik}: Multiscale graph neural network autoencoders for interpretable scientific machine learning, {\it Journal of Computational Physics}, Volume 495, 15 December 2023, 112537.

\item C. Moss, M. Puccioni, \textbf{R. Maulik}, C. Jacquet, D. Apgar, G. V. Iungo : Predicting Wind Farm Operations with Machine Learning and the P2D-RANS model: A Case Study for an AWAKEN Site, {\it Wind Energy}, 2023.

\item S. Lin, G. Mengaldo, \textbf{R. Maulik}: Online data-driven changepoint detection for high-dimensional dynamical systems, {\it Chaos: An Interdisciplinary Journal of Nonlinear Science}, 33, 10, 2023.

\item J. Botvinick-Greenhouse, Y. Yang, \textbf{R. Maulik}: Generative Modeling of Time-Dependent Densities via Optimal Transport and Projection Pursuit, {\it Chaos: An Interdisciplinary Journal of Nonlinear Science}, 33, 10, 2023.

\item C. Jang, J. Choi, \textbf{R. Maulik}, D. Lim: Determinants of Adult Education and Training Participation in the United States: A Machine Learning Approach, {\it Adult Education Quarterly}, 2023.

\item \textbf{R. Maulik}, R. Egele, K. Raghavan, P. Balaprakash: Quantifying uncertainty for deep learning based forecasting and flow-reconstruction using neural architecture search ensembles, {\it Physica D.}, 454, 133852, 2023.

\item A. Aygun, \textbf{R. Maulik}, A. Karakus: Physics-Informed Neural Networks for Mesh Deformation with Exact Boundary Enforcement, {\it Engineering Applications of Artificial Intelligence}, 125, 106660, 2023.

\item S. Bhola, S. Pawar, P. Balaprakash, \textbf{R. Maulik}: Multi-fidelity reinforcement learning framework for shape optimization, {\it Journal of Computational Physics}, 482, 112018, 2023.

\item V. Shankar, V. Puri, R. Balakrishnan, \textbf{R. Maulik}, V. Vishwanathan: Differentiable physics-enabled closure modeling for Burgers' turbulence, {\it Machine Learning Science and Technology}, 4, 015017, 2023.

\item J. Choi, W. Wehde, \textbf{R. Maulik}: Politics of Problem Definition: Comparing Public Support of Climate Change Mitigation Policies using Machine Learning, {\it Review of Policy Research}, 2022.

\item S. Mondal, G. Magnotti, B. Lusch, \textbf{R. Maulik}, R. Torelli: Machine Learning-Enabled Prediction of Transient Injection Map in Automotive Injectors With Uncertainty Quantification, {\it Journal of Engineering for Gas Turbines and Power}, 145(4), 041015, 2023

\item A. Linot, J. Burby, Q. Tang, P. Balaprakash, M. Graham, \textbf{R. Maulik}: Stabilized Neural Ordinary Differential Equations for Long-Time Forecasting of Dynamical Systems, {\it Journal of Computational Physics}, 474, 111838, 2022.

\item N. Garland, \textbf{R. Maulik}, Q. Tang, X. Tang, P. Balaprakash, Efficient training of artificial neural network surrogates for a collisional-radiative model through adaptive parameter space sampling, {\it Machine Learning Science and Technology}, 3 (4), 045003, 2022.

\item M. Morimoto, K. Fukami, \textbf{R. Maulik}, R. Vinuesa, K. Fukagata: Assessments of model-form uncertainty using Gaussian stochastic weight averaging for fluid-flow regression, {\it Physica D: Nonlinear Phenomena}, 133454, 2022.

\item A. Lario, \textbf{R. Maulik}, G. Rozza, G. Mengaldo: Neural-network learning of SPOD dynamics, {\it Journal of Computational Physics}, 111475, 2022.

\item G. Iungo, \textbf{R. Maulik}, S. Renganathan, S. Letizia: Machine-learning identification of the variability of mean velocity and turbulence intensity for wakes generated by onshore wind turbines: Cluster analysis of wind LiDAR measurements, {\it Journal of Renewable and Sustainable Energy}, 14 (Cover article), 023307, 2022.

\item \textbf{R. Maulik}, V. Rao, J. Wang, G. Mengaldo, E. Constantinescu, B. Lusch, P. Balaprakash, I. Foster, R. Kotamarthi, Efficient high-dimensional variational data assimilation with machine-learned reduced-order models, {\it Geoscientific Model Development,} 15, 3433–3445, 2022..

\item Y. Lu, \textbf{R. Maulik}, T. Gao, F. Dietrich, I. Kevrekidis, J. Duan: Learning the temporal evolution of multivariate densities via normalizing flows, {\it Chaos: An Interdisciplinary Journal of Nonlinear Science}, 32 (3), 033121, 2022.

\item \textbf{R. Maulik}, D. Fytanidis, B. Lusch, S. Patel, V. Vishwanath: PythonFOAM: In-situ data analyses with OpenFOAM and Python, {\it Journal of Computational Science}, 62, 101750, 2022.

\item S. Renganathan, \textbf{R. Maulik}, G. Iungo, S. Letizia: Data-driven wind turbine wake modeling using probabilistic machine learning, {\it Neural Computing and Applications}, 34, 6171–618, 2022.

\item K. Lyras, \textbf{R. Maulik}, D. Schmidt: Machine-learning accelerated turbulence modelling of transient flashing jets, {\it Physics of Fluids}, 33, 127104 (2021).

\item K. Fukami, \textbf{R. Maulik}, N. Ramachandra, K. Taira, K. Fukagata: Global field reconstruction from sparse sensors with Voronoi tessellation-assisted deep learning, {\it Nature Machine Intelligence}, 2021.

\item B. Hamzi, \textbf{R. Maulik}, H. Owhadi: Simple, low-cost, and accurate, data-driven geophysical forecasting with learned kernels, {\it Proceedings of the Royal Society A}, 477: 20210326, 2021.

\item G. Mengaldo, \textbf{R. Maulik}, PySPOD: A Python package for Spectral Proper Orthogonal Decomposition (SPOD), {\it Journal of Open Source Software}, 6 (60), 2862, 2021.

\item \textbf{R. Maulik}, B. Lusch, P. Balaprakash: Reduced-order modeling of advection-dominated systems with recurrent neural networks and convolutional autoencoders , {\it Physics of Fluids}, 33, 037106, 2021.

\item S. Pawar, \textbf{R. Maulik}: Distributed deep reinforcement learning for simulation control, {\it Machine Learning: Science and Technology}, 2, 025029, 2021.

\item S. Renganathan, \textbf{R. Maulik}, J. Ahuja: Enhanced data efficiency using deep neural networks and Gaussian processes for aerodynamic design optimization, {\it Aerospace Science and Technology}, 111, 106522, 2021.

\item J. Burby, Q. Tang, \textbf{R. Maulik}: Fast neural Poincar\'{e} maps for toroidal magnetic fields, {\it Plasma Physics and Controlled Fusion}, 63, 024001, 2021.

\item \textbf{R. Maulik}, T. Botsas, N. Ramachandra, M. Lachlan, I. Pan: Latent-space time evolution of non-intrusive reduced-order models using Gaussian process emulation, {\it Physica D: Nonlinear Phenomena}, 132797, 2021. 

\item \textbf{R. Maulik}, H. Sharma, S. Patel, B. Lusch, E. Jennings : A turbulent eddy-viscosity surrogate modeling framework for Reynolds-Averaged Navier-Stokes simulations, {\it Computers and Fluids}, 104777, 2020. 

\item \textbf{R. Maulik}, K. Fukami, N. Ramachandra, K. Fukagata, K. Taira : Probabilistic neural networks for fluid flow surrogate modeling and data recovery, {\it Physical Review Fluids}, 5, 104401, 2020. 

\item \textbf{R. Maulik}, P. Balaprakash, B. Lusch: Non-autoregressive time-series methods for stable parameteric reduced-order models, {\it Physics of Fluids}, 32, 087115, 2020.

\item \textbf{R. Maulik}, N. Garland, X. Tang, P. Balaprakash: Neural network representability of fully ionized plasma fluid model closures, {\it Physics of Plasmas}, 27, 072106, 2020.

\item J. Choi, S. Robinson, \textbf{R. Maulik}, W. Wehde: What Matters the Most for Individual Disaster Preparedness? Understanding Emergency Preparedness Using Machine Learning, {\it Natural Hazards}, 103, 1183-1200, 2020. 

\item S. Renganathan \textbf{R. Maulik}, V. Rao : Machine learning for nonintrusive model order reduction of the parametric inviscid transonic flow past an airfoil, {\it Physics of Fluids}, 32, 047110, 2020. 

\item \textbf{R. Maulik}, O. San: Numerical assessments of a parametric implicit large eddy simulation model, {\it Journal of Computational and Applied Mathematics}, 112866, 2020.

\item \textbf{R. Maulik}, O. San, J. Jacob: Spatiotemporally dynamic implicit large eddy simulation using machine learning classifiers, {\it Physica D: Nonlinear Phenomena}, 406, 132409, 2020. 

\item \textbf{R. Maulik}, A. Mohan, B. Lusch, S. Madireddy, P. Balaprakash, D. Livescu: Time-series learning of latent-space dynamics for reduced-order model closure, {\it Physica D: Nonlinear Phenomena}, 405, 132368, 2020. 

\item Y. Hossain, \textbf{R. Maulik}, H. Park, M. Ahmed, C. Bach, O. San: Improvement of Unitary Equipment and Heat Exchanger Testing Methods, {\it ASHRAE Transactions}, 125.2, 2019. 

\item \textbf{R. Maulik}, O. San, J. Jacob, C. Crick: Sub-grid scale model classification and blending through deep learning, {\it Journal of Fluid Mechanics}, 870, 784-812, 2019. 

\item O. San, \textbf{R. Maulik}, M. Ahmed: An artificial neural network framework for reduced order modeling of transient flows, {\it Communications in Nonlinear Science and Numerical Simulation}, 77, 271-287, 2019. 

\item \textbf{R. Maulik}, O. San, A. Rasheed, P. Vedula: Subgrid modeling for two-dimensional turbulence using artificial neural networks, {\it Journal of Fluid Mechanics}, 858, 122-144, 2019. 

\item \textbf{R. Maulik}, O. San, A. Rasheed, P. Vedula: Data-driven deconvolution for large eddy simulation of Kraichnan turbulence, {\it Physics of Fluids}, 30, 125109, 2018. 

\item O. San, \textbf{R. Maulik}: Stratified Kelvin-Helmholtz turbulence of compressible shear flows, {\it Nonlinear Processes in Geophysics}, 25, 457--476, 2018.

\item O. San, \textbf{R. Maulik}: Extreme learning machine for reduced order modeling of turbulent geophysical flows, {\it Physical Review E}, 97, 042322, 2018. 

\item O. San, \textbf{R. Maulik}: Machine learning closures for model order reduction of thermal fluids,  {\it Applied Mathematical Modelling}, 60, 681-710, 2018. 

\item \textbf{R. Maulik}, O. San, R. Behera : An adaptive multilevel wavelet framework for scale-selective WENO reconstruction schemes, {\it International Journal of Numerical Methods in Fluids}, 87 (5), 239-269, 2018. 

\item O. San, \textbf{R. Maulik}: Neural network closure models for nonlinear model order reduction, {\it Advances in Computational Mathematics}, 44, 1717-1750, 2018. 

\item \textbf{R. Maulik}, O. San: A dynamic closure modeling framework for large eddy simulation using approximate deconvolution: Burgers equation, {\it Cogent Physics}, 5, 1464368, 2018. 

\item \textbf{R. Maulik}, O. San: A neural network approach for the blind deconvolution of turbulent flows, {\it Journal of Fluid Mechanics}, 831, 151-181, 2017. 

\item \textbf{R. Maulik}, O. San: A novel dynamic framework for subgrid-scale parametrization of mesoscale eddies in quasigeostrophic turbulent flows, {\it Computers and Mathematics with Applications}, 74, 420-445, 2017. 

\item \textbf{R. Maulik}, O. San: Explicit and implicit LES closures for Burgers turbulence, {\it Journal of Computational and Applied Mathematics}, 327, 12-40, 2017. 

\item \textbf{R. Maulik}, O. San: Resolution and energy dissipation characteristics of implicit LES and explicit filtering models for compressible turbulence, {\it Fluids}, 2(2)-14, 2017. 

\item \textbf{R. Maulik}, O. San: A dynamic subgrid-scale modeling framework for Boussinesq turbulence, {\it International Journal of Heat and Mass Transfer}, 108, 1656-1675, 2017. 

\item \textbf{R. Maulik}, O. San: A dynamic framework for functional parameterisations of the eddy viscosity coefficient in two-dimensional turbulence, {\it International Journal of Computational Fluid Dynamics}, 31(2), 69-92, 2017. 

\item \textbf{R. Maulik}, O. San: A stable and scale-aware dynamic modeling framework for subgrid-scale parameterizations of two-dimensional turbulence, {\it Computers \& Fluids} 158, 11-38, 2016. 

\item \textbf{R. Maulik}, O. San: Dynamic modeling of the horizontal eddy viscosity coefficient for quasigeostrophic ocean circulation problems, {\it Journal of Ocean Engineering and Science} 1, 300-324, 2016.

\item H. H. Marbini, \textbf{R. Maulik}: A biphasic transversely isotropic poroviscoelastic model for the unconfined compression of hydrated soft tissue, {\it Journal of Biomechanical Engineering} 138, 031003, 2016.

\end{enumerate}

\subsection*{Peer-reviewed conference publications}

\begin{enumerate}

\item H. Zhang, Y. Liu, \textbf{R. Maulik}: Semi-Implicit Neural Ordinary Differential Equations, The 39th Annual AAAI Conference on Artificial Intelligence, 2025. \url{https://openreview.net/forum?id=MeALSxFb5A&noteId=ffoAQc9SLo}

\item D. Chakraborty, S. Chung, A. Chattopadhyay, \textbf{R. Maulik}: Improved deep learning of chaotic dynamical systems with multistep penalty losses, Conference on Applied AI and Scientific Machine Learning, IISC Bangalore (Spotlight paper $<5\%$) {\it arXiv:2410.05572}.

\item T. Nguyen, R. Shah, H. Bansal, T. Arcomano, S. Madireddy, \textbf{R. Maulik}, V. Kotamarthi, I. Foster, A. Grover: Scaling transformers for skillful and reliable medium-range weather forecasting, {\it Neural Information Processing Systems, 2024}

\item R. Egele, \textbf{R. Maulik}, K. Raghavan, P. Balaprakash, B. Lusch: AutoDEUQ: Automated Deep Ensemble with Uncertainty Quantification, {\it 26th International Conference on Pattern Recognition}, 1908--1914, 2022.

\item Sudeepta Mondal, Gina M. Magnotti, Bethany Lusch, \textbf{Romit Maulik}, Roberto Torelli: Machine Learning-Enabled Prediction of Transient Injection Map in Automotive Injectors With Uncertainty Quantification, ASME Internal Combustion Engine Fall Conference, 2021. 

\item H. Shan, Y. Sun, \textbf{R. Maulik}, T. Xu: Application of Artificial Neural Network in the APS LINAC Bunch Charge Transmission Efficiency, {\it 12th International Particle Accelerator Conference (IPAC), 2021.}, \\ \texttt{https://accelconf.web.cern.ch/ipac2021/papers/tupab287.pdf}.

\item V. Sastry, \textbf{R. Maulik}, V. Rao, B. Lusch, S. Renganathan, R. Kotamarthi: Data-driven deep learning emulators for geophysical forecasting, {\it International Conference on Computational Science, 433-446, 2021}, \\ \texttt{https://doi.org/10.1007/978-3-030-77977-1\_35}. 

\item \textbf{R. Maulik}, R. Egele, B. Lusch,  P. Balaprakash: Recurrent neural network architecture search for geophysical emulation, {\it Proceedings of the International Conference for High Performance Computing, Networking, Storage and Analysis (SC), 2020, 10.5555/3433701.3433711}. 

\item V. Rao, \textbf{R. Maulik}, E. Constantinescu, M. Anitescu: A machine learning method for computing rare event probabilities, {\it International Conference on Computational Science, 169-182, 2020}, \\ \texttt{https://link.springer.com/chapter/10.1007\%2F978-3-030-50433-5\_14}. 

\item \textbf{R. Maulik}, O. San, C. Bach: A computational investigation of the effect of ground clearance in vertical ducting systems, International High Performance Buildings Conference, Herrick Laboratories, Purdue University, 2018. \\ \texttt{https://docs.lib.purdue.edu/ihpbc/308/}.

\end{enumerate}

\subsection*{Conference publications}

\begin{enumerate}

\item \textbf{R. Maulik}, H. Sharma, S. Patel, B. Lusch, E. Jennings: Deploying deep learning in OpenFOAM with TensorFlow: A tutorial, AIAA SciTech Forum 2021, \texttt{https://doi.org/10.2514/6.2021-1485}.

\item P. Milan, R. Torelli, B. Lusch, \textbf{R. Maulik}, G. Magnotti: Data-Driven Modeling of Large-Eddy Simulations for Fuel Injector Design, AIAA SciTech Forum 2021, \\ \texttt{https://doi.org/10.2514/6.2021-1016}.

\item \textbf{R. Maulik}, V. Rao, S. Renganathan, S. Letizia, G. Iungo: Cluster analysis of wind turbine wakes measured through a scanning Doppler wind LiDAR, AIAA SciTech Forum 2021, \\ \texttt{https://doi.org/10.2514/6.2021-1181}.

\end{enumerate}

\subsection*{Peer-reviewed workshop proceedings}

\begin{enumerate}

\item X. Li, \textbf{R. Maulik}: Reinforcement Learning Stability Analysis in the Latent Space of Actions, AAAI 2025 bridge program on Knowledge-guided Machine Learning, February 25-26, 2025.

\item J. Henry, \textbf{R. Maulik}: Reduced complexity modeling of fluidic oscillators with data-driven boundary conditions, AAAI 2025 bridge program on Knowledge-guided Machine Learning, February 25-26, 2025.

\item H. Guan, D. Chakraborty, \textbf{R. Maulik}: Atmospheric Super-Resolution with Neural Operators, AAAI 2025 bridge program on Knowledge-guided Machine Learning, February 25-26, 2025.

\item D. Chakraborty, S. W. Chung, A. Chattopadhyay, T. Arcomano, \textbf{R. Maulik}: Improved deep learning of chaotic dynamical systems with multistep penalty losses, AAAI 2025 bridge program on Knowledge-guided Machine Learning, February 25-26, 2025.

\item H. Kim,  S. Barwey, \textbf{R. Maulik}: Scalable, adaptive, and explainable scientific machine learning with applications to surrogate models of partial differential equations, , AAAI 2025 bridge program on Knowledge-guided Machine Learning, February 25-26, 2025.

\item H. Kim, S. Barwey, \textbf{R. Maulik}: Scalable, adaptive, and explainable scientific machine learning with applications to surrogate models of partial differential equations, AAAI Symposium on Integrated Approaches to Computational Scientific Discovery, November 7-9, 2024.

\item S. Barwey, R. Balin, B. Lusch, S. Patel, R. Balakrishnan, P. Pal, \textbf{R. Maulik}, V. Vishwanath: Scalable and Consistent Graph Neural Networks for Distributed Mesh-based Data-driven Modeling, (Workshop on Machine Learning with Graphs in High Performance Computing Environments), Supercomputing 2024.

\item H. Guan, T. Arcomano, A. Chattopadhyay, \textbf{R. Maulik}: LUCIE: A Lightweight Uncoupled ClImate Emulator with long-term stability and physical consistency for O(1000)-member ensembles, (Machine Learning for Earth System Modeling Workshop, ICML 2024).

\item T. Nguyen, R. Shah, H. Bansal, T. Arcomano, S. Madireddy, \textbf{R. Maulik}, V. Kotamarthi, I. Foster, A. Grover: Scaling transformers for skillful and reliable medium-range weather forecasting, ICLR AI4DiffEqtnsInSci Workshop 2024 (Accepted as spotlight presentation and awarded best paper).

\item A. Nair, S. Barwey, P. Pal, \textbf{R. Maulik}: Investigation of Latent Time-Scales in Neural ODE Surrogate Models, ICLR AI4DiffEqtnsInSci Workshop 2024 (Accepted as poster).

\item H. Zhang, Y. Liu, \textbf{R. Maulik} Semi-Implicit Neural Ordinary Differential Equations for Learning Chaotic Systems, NeurIPS 2023 Workshop Heavy Tails in Machine Learning, NeurIPS 2023.

\item V. Shankar, S. Barwey, \textbf{R. Maulik}, V. Vishwanathan: Practical implications of equivariant and invariant graph neural networks for fluid flow modeling, Physics4ML Workshop, ICLR 2023, \url{https://openreview.net/forum?id=3Y6XRCIUT5}.

\item \textbf{R. Maulik}, G. Mengaldo: PyParSVD: A streaming, distributed and randomized singular-value-decomposition library, (7th International Workshop on Data Analysis and Reduction for Big Scientific Data (DRBSD-7), Supercomputing 2021).

\item D. Skinner, \textbf{R. Maulik}: Meta-modeling strategy for data-driven forecasting, \textit{Tackling Climate Change with Machine Learning Workshop, NeurIPS}, 2020. \url{https://www.climatechange.ai/papers/neurips2020/13.html}. 

\item N. Garland, \textbf{R. Maulik}, Q. Tang, X. Tang, P. Balaprakash: Progress towards high fidelity collisional-radiative model surrogates for rapid in-situ evaluation, \textit{Machine Learning for Physical Sciences Workshop, NeurIPS}, 2020. \url{https://ml4physicalsciences.github.io/2020/files/NeurIPS_ML4PS_2020_79.pdf}.

\item K. Fukami, \textbf{R. Maulik}, N. Ramachandra, K. Fukagata, K. Taira: Probabilistic neural network-based reduced-order surrogate for fluid flows, \textit{Machine Learning for Physical Sciences Workshop, NeurIPS}, 2020. \url{https://ml4physicalsciences.github.io/2020/files/NeurIPS_ML4PS_2020_7.pdf}

\item \textbf{R. Maulik}, R. S. Assary, P. Balaprakash: Site-specific graph neural network for predicting protonation energy of oxygenate molecules, {\it Machine Learning for Physical Sciences Workshop, NeurIPS}, 2019. \url{https://ml4physicalsciences.github.io/2019/files/NeurIPS_ML4PS_2019_134.pdf}

\item \textbf{R. Maulik}, V.Rao, S. Madireddy, B. Lusch, P. Balaprakash: Using recurrent neural networks for nonlinear component computation in advection-dominated reduced-order models, \textit{Machine Learning for Physical Sciences Workshop, NeurIPS}, 2019. \url{https://ml4physicalsciences.github.io/2019/files/NeurIPS_ML4PS_2019_99.pdf}.

\end{enumerate}



\section*{Talks presented}

\begin{enumerate}

\item Scientific machine learning for forecasting dynamical systems exhibiting deterministic chaos, Summer School: Scientific Machine Learning, Brin Mathematics Research Center, University of Maryland, August 2025.

\item PDE-Constrained Learning of Data-Driven Turbulence Models, \textbf{Invited talk}, Thermal Transport Caf\'{e}, MIT, March 13, 2025.

\item Differentiable Physics: A physics-constrained and data-driven paradigm for scientific discovery, \textbf{Invited talk}, Penn State University Computational and Applied Mathematics Colloquium, Jan 27, 2025.

\item Non-intrusive reduced-order modeling with quantum reservoir computing, \textbf{Invited talk}, IACM Digital Twins in Engineering and ECCOMAS Artificial Intelligence and Computational Methods in Applied Science, February 2025, Paris, France.

\item Deep-learning enhanced data assimilation for digital twins of high-dimensional complex dynamical systems, \textbf{Invited talk}, Korean Atomic Energy Research Institute (KAERI) Annual Workshop, January 22, 2025.

\item Uncertainty quantification for scientific machine learning with joint neural architecture and hyperparameter search with DeepHyper, \textbf{Invited talk}, Banff International Research Station workshop on Uncertainty Quantification in Neural Network Models, February 2025, Banff, Canada.

\item Neural ordinary differential equations for scientific machine learning, \textbf{Invited talk}, Param-Intelligence (PI) Seminar Series, Worcester Polytechnic Institute, November 7, 2024.

\item Scalable, adaptive, and explainable geometric deep learning with applications to computational physics, \textbf{Invited talk}, University of Illinois at Chicago, September 20, 2024.

\item PDE-constrained machine learning of data-driven closure models, \textbf{Invited talk}, Argonne National Laboratory, September 19, 2024.

\item Neural ordinary differential equations for scientific machine learning, \textbf{Invited talk}, AIGrid Hackathon, Rugen Island, Germany, September 12, 2024.

\item PDE-constrained machine learning of data-driven closure models, \textbf{Invited talk}, Society of Petroleum Engineering CFD Seminar, August 22, 2024.

\item Differentiable physics for closure modeling from data, \textbf{Invited talk}, ONR Erosion Workshop, Johns Hopkins University, August 14, 2024.

\item Interpretable Fine-Tuning and Error Indication for Graph Neural Network Surrogate Models, USACM UQ-MLIP, Washington DC, August 13, 2024.

\item Interpretable SciML for fluid flows using multiscale graph neural networks, \textbf{Invited talk}, Healthy People and Healthy Planet Workshop, Imperial College, July 25, 2024.

\item Scalable, adaptive, and explainable scientific machine learning with applications to computational fluid dynamics, \textbf{Invited talk}, Hong Kong University of Science and Technology Mechanical Engineering Seminar Series, April 17, 2024.

\item Differentiable physics for turbulence closure modeling from data, \textbf{Invited talk}, Unravel project meeting, University of Eindhoven, April 17, 2024.

\item Scalable, adaptive, and explainable scientific machine learning with applications to computational fluid dynamics, \textbf{Guest lecture}, NUEN 689 ``Deep Learning for Engineering Applications", Texas A \& M University, April 9, 2024.

\item Differentiable physics for turbulence closure modeling from data, \textbf{Invited talk}, University of Maryland Numerical Analysis Seminar Series, April 2, 2024.

\item Scalable, adaptive, and explainable scientific machine learning with applications to computational fluid dynamics, \textbf{Invited talk}, Fluid Dynamics Research Consortium Seminar Series, Pennsylvania State University, March 28, 2024.

\item Turbulence modeling for large-eddy simulations using neural differential equations, \textbf{Invited talk}, International Conference on Differential Equations for Data Science 2024 (DEDS2024), Feb 19-21, 2024.

\item Scalable, adaptive, and explainable scientific machine learning with applications to computational fluid dynamics, \textbf{Invited talk}, Advanced Modeling and Simulations Seminar Series, University of Texas, El Paso, Feb 9, 2024.

\item On the construction of data-driven closures for large eddy simulations of turbulence, \textbf{Invited talk}, ERCOFTAC Workshop on Machine Learning for Fluid Dynamics, Sorbonne University, Paris, France, 6-8 March 2024.

\item Data assimilation with scientific machine learning, \textbf{Invited talk}, Indian Institute of Science, Education, and Research, Pune, December, 2023.

\item Anomaly detection for dynamical systems using Bayesian online changepoint detection, \textbf{Invited talk}, Brij Disa Center for Data Science and AI, Indian Institute of Management, Ahmedabad, December 2023.

\item Multiscale Graph Neural Network Architectures for Interpretable Scientific Machine Learning, \textbf{Invited talk}, CMAI Seminar Series, George Mason University, October, 2023.

\item The research program of the Interdisciplinary Scientific Computing Laboratory, \textbf{Invited talk}, AI/ML Technical Group, Department of Aerospace Engineering, Pennsylvania State University, October, 2023.

\item Differentiable turbulence modeling, \textbf{Invited talk}, Aerospace Seminar Series, TU Delft, October, 2023.

\item Differentiable turbulence modeling, \textbf{Invited talk}, Autumn school on Scientific Machine Learning, CWI Amsterdam, October, 2023.

\item PythonFoam: In-situ data analyses with OpenFOAM and Python, \textbf{Invited talk}, MS 421.1, ``Software Tools for Uncertainty Quantification and Machine Learning with Applications to Computational Science", 17th U.S. National Congress on Computational Mechanics, Albuquerque, 2023.

\item Multiscale Graph Neural Network Autoencoders for Interpretable Scientific Machine Learning, \textbf{Invited talk}, BIRS Scientific Machine Learning workshop, Banff International Research Station, June, 2023.

\item Neural architecture search for scientific machine learning, \textbf{Invited talk}, ICERM topical workshop on Mathematical and Scientific Machine Learning, Brown University, June, 2023.

\item Multiscale graph neural networks for scalable surrogate modeling of fluid dynamical systems, \textbf{Invited talk}, International Workshop on Reduced Order Methods, National University of Singapore, May, 2023.

\item Breaking boundaries: Why interdisciplinary research is key for tackling grand challenges, \textbf{Plenary talk}, Oklahoma State University, Mechanical and Aerospace Engineering Graduate Search Symposium, March 24, 2023.

\item A stabilized neural ordinary differential equation for scientific machine learning, \textbf{Invited talk}, SIAM Conference on Computational Science and Engineering, Amsterdam, 2023.

\item Neural architecture search for scientific machine learning, \textbf{Invited talk}, Rutgers Efficient AI Seminar Series, February, 2023.

\item Accelerating scientific discovery using physics-informed machine learning, \textbf{Invited talk}, Machine Learning for e-Science, Swedish e-Science Research Centre, November 30, 2022.

\item Efficient high-dimensional variational data assimilation with machine-learned reduced-order models, Bulletin of the American Physical Society, Division of Fluid Dynamics, November, 2022.

\item A stabilized neural ordinary differential equation for scientific machine learning, \textbf{Invited talk}, SIAM Mathematics of Data Science, San Diego, September 28, 2022.

\item Neural forecasting of high-dimensional dynamical systems, \textbf{Invited talk}, University of Pittsburgh Computational Mathematics Seminar, September 20, 2022.

\item Quantifying Uncertainty in Deep Learning for Fluid Flow Reconstruction, \textbf{Invited talk}, USACM Thematic Conference on Uncertainty Quantification for Machine Learning Integrated Physics Modeling (MLIP), Crystal City, Virginia, August 18-19, 2022.

\item A stabilized neural ordinary differential equation for scientific machine learning, \textbf{Invited talk}, Argonne Training Program on Exascale Computing (ATPESC), August 12, 2022.

\item Learning nonlinear dynamical systems from data using scientific machine learning, \textbf{Invited talk}, 2022 AI + Science Summer School, University of Chicago, Data Sciences Institute, August 9, 2022.

\item Non-intrusive reduced-order modeling using scientific machine learning, \textbf{Invited talk}, Summer School on Reduced Order Methods in CFD, SISSA Trieste, July 13, 2022.

\item Learning nonlinear dynamical systems from data using scientific machine learning, \textbf{Invited talk}, Accurate ROMs for Industrial Applications, Virginia Tech, July 7, 2022.

\item Learning nonlinear dynamical systems from data using scientific machine learning, \textbf{Invited talk}, Brown University CRUNCH Seminar series, May 27, 2022. 

\item Simple, low-cost and accurate data-driven geophysical forecasting with learned kernels, \textbf{Invited talk}, SIAM UQ, Atlanta, Georgia, April 12, 2022.

\item Reduced-order modeling of high-dimensional dynamical systems using scientific machine learning, \textbf{Invited talk}, IBiM Seminar Series, March 2022.

\item Emulating nonlinear dynamical systems from data using scientific machine learning, \textbf{Invited talk}, APS March Meetings, March 14, 2022.

\item Reduced-order modeling of high-dimensional dynamical systems using scientific machine learning, \textbf{Invited talk}, National University of Singapore, Department of Mechanical Engineering, Distinguished Seminar Series, February 10, 2022.

\item Emulating complex systems from data using scientific machine learning, \textbf{Invited talk}, North Carolina State University, Department of Mathematics, February 1, 2022.

\item Research at the intersection of mathematics, computation, and data, \textbf{Invited webinar}, BIT Mesra Alumni Association North America Faculty Webinar, January 29, 2022.

\item Reduced-order modeling of high-dimensional systems using scientific machine learning, \textbf{Invited talk}, University of Waterloo, Department of Applied Mathematics, January 18, 2022.

\item Research at the intersection of mathematics, computation, and data, \textbf{Invited talk}, Physics \& Engineering speaker series, North Park University, November 17, 2021.

\item Reduced-order modeling of high-dimensional systems using scientific machine learning, \textbf{Invited talk}, 2021 CBE Computing Seminar Series, University of Wisconsin-Madison, October 22, 2021.

\item Parallelized emulator discovery and uncertainty quantification using DeepHyper, Mechanistic Machine Learning and Digital Twins for Computational Science, Engineering \& Technology, September 26-29, 2021.

\item Modified neural ordinary differential equations for stable learning of chaotic dynamics, \textbf{Invited talk}, Applied Mathematics Seminar Series, Texas Tech University, September 1, 2021.

\item Neural architecture search for surrogate modeling, \textbf{Invited talk}, ML4I Forum, Lawrence Livermore National Laboratory, August 10-12, 2021.

\item Scalable scientific machine learning for computational fluid dynamics, \textbf{Plenary talk}, Computational Sciences and AI in Industry, June 7-9, 2021.

\item Neural architecture search for surrogate modeling, \textbf{Invited talk}, DDPS Seminar Series, Lawrence Livermore National Laboratory, May 27, 2021.

\item Incorporating inductive biases for the surrogate modeling of dynamical systems, \textbf{Invited talk} at Machine Learning for Dynamical Systems Special Interest Group, Alan Turing Institute, Imperial College London, April 14, 2021.

\item Surrogate modeling with learned kernels (Kernel Flows), \textbf{Invited talk}, Uncertainty Quantification in Climate Science, NASA JPL Climate Center Virtual Workshop, March 24, 2021.

\item Scalable recurrent neural architecture search for geophysical emulation, \textbf{Invited talk} at SIAM-CSE Minisymposium on Physics-Guided Machine Learning and Data-Driven Methods in Computational Geoscience, 2021.

\item Incorporating Inductive Biases as Hard Constraints for Scientific Machine Learning, MCS-LANS seminar, Argonne National Laboratory, February 2021.

\item Scalable scientific machine learning for computational fluid dynamics, \textbf{Invited talk}, Department of Mechanical Engineering, The City College of New York, October, 2020.

\item Data-driven model order reduction for geophysical emulation. \textbf{Invited talk} at the Second Symposium on Machine Learning and Dynamical Systems, Fields Institute, Toronto, September, 2020.

\item Scalable scientific machine learning for computational fluid dynamics, \textbf{Invited talk}, Department of Mechanical Engineering, Rice University, September, 2020.

\item Machine Learning Enablers for System Optimization and Design, MCS-LANS seminar, Argonne National Laboratory, August 2020.

\item Non-intrusive reduced-order model search for geophysical emulation, \textbf{Guest lecture}, MAE259a: Data science for fluid dynamics (offered by Kunihiko Taira), University of California Los Angeles, June 2020.

\item Spatiotemporally dynamic implicit large eddy simulation using machine learning classifiers, Session on Domain-Aware, Interpretable and Robust Scientific Machine Learning Methods Applied to Computational Mechanics, AIAA Aviation Forum, June, 2020, Reno. 

\item Machine learning for computational fluid dynamics, \textbf{Invited talk} at PyData Meetup Chicago, May 2020.

\item Recurrent neural architecture search for geophysical emulation using DeepHyper, \textbf{Invited talk} at AI-HPC seminar, Argonne National Laboratory, April, 2020.

\item Machine Learned Reduced-Order Models for Advective Partial Differential Equations, MCS-LANS seminar, Argonne National Laboratory, February, 2020.

\item Machine Learned Reduced-Order Models for Advective Partial Differential Equations, \textbf{Invited talk}, 2020 Spring Multiscale Seminar, Illinois Institute of Technology, Chicago, February, 2020.

\item General purpose data science for general purpose CFD: Integrating Tensorflow into OpenFOAM at scale, \textbf{Invited poster}, Workshop for Machine Learning for Transport Phenomena, February, 2020, Dallas.

\item Machine learning of sequential data for non-intrusive reduced-order models, Bulletin of the American Physical Society, Division of Fluid Dynamics, November, 2019.

\item Tackling the limitations of conventional ROMs for advection-dominated nonlinear dynamical systems using machine learning, \textbf{Invited talk}, Advanced Statistics meets Machine Learning-III workshop, Argonne National Laboratory, November, 2019.

\item Data-driven sub-grid models for the large-eddy simulation of turbulence, \textbf{Invited talk}, John Zink Hamworthy Combustion, Tulsa, August, 2019.

\item Data-driven deconvolution for the sub-grid modeling of large eddy simulations of two-dimensional turbulence, SIAM-CSE, March, 2019.

\item Data-driven deconvolution for the large eddy simulation of Kraichnan turbulence, Bulletin of the American Physical Society, Division of Fluid Dynamics, November, 2018.

\item A computational investigation of the effect of ground clearance in vertical ducting systems, 2018, Purdue University, Herrick Labs Conferences, July, 2018. 

\item A neural network approach for the blind deconvolution of turbulent flows, Bulletin of the American Physical Society, Division of Fluid Dynamics, November, 2017.

\item A generalized wavelet based grid-adaptive and scale-selective implementation of WENO schemes for conservation laws, Texas Applied Mathematics and Engineering Symposium, The University of Texas, Austin, September 2017.

\item An explicit filtering framework based on Perona-Malik anisotropic diffusion for shock capturing and subgrid scale modeling of Burgers' turbulence, Bulletin of the American Physical Society, Division of Fluid Dynamics, November, 2016.

\item A dynamic hybrid subgrid-scale modeling framework for large eddy simulations, Bulletin of the American Physical Society, Division of Fluid Dynamics, November, 2016.

\end{enumerate}

\section*{Professional service}

\subsection*{Committee membership}

\begin{itemize}
\item PythonFOAM Workshop Lead Organizer (https://www.alcf.anl.gov/events/alcf-pythonfoam-workshop).
\item DOE INCITE program (2020) - Reviewed 2 proposals every year
\item ADSP program (2020) - Reviewed 2 proposals every year
\item International Conference on Parallel Processing, Chicago, 2021 (Reviewed 6 articles).
\item Wilkinson Postdoctoral Fellowship Committee, MCS Division, Argonne National Laboratory, 2022.
\item DOE AI for Earth System Predictability workshop session chair for neural networks.
\end{itemize}

\subsection*{Tutorials organized}

\begin{itemize}
  \item Tutorial lead - Autoencoders for PDE surrogate models, ATPESC 2020.
  \item Tutorial lead - Statistical methods for machine learning, ALCF AI4Science tutorial 2019, Argonne National Laboratory.
  \item Tutorial lead - DeepHyper for scalable hyperparameter and neural architecture search on ALCF machines, ALCF Simulation Data and Learning workshop 2019, Argonne National Laboratory.
  \item TensorFlow workshop, Mechanical \& Aerospace Engineering, Oklahoma State University, 2018.
  \item An introduction to high performance computing for middle school kids, National Lab Day, Oklahoma State University 2017, 2018.
\end{itemize}

\subsection*{Minisymposia}

\begin{itemize}
  \item MS Organizer \& Session chair, 16th, 17th USNCCM, SIAM-CSE 2019, 2021, 2023, AIAA Aviation 2020, 2024.
  \item Co-organizer - Argonne National Laboratory - AI, Statistics and Machine Learning Journal Club.
  \item Session chair - MAE Graduate Research Symposium, Oklahoma State University, 2018.
\end{itemize}

\subsection*{Journal Review}

\begin{itemize}
  \item Editor - Results in Engineering, Elsevier.
  \item Journal reviewer for - AIAA Journal, Applied Mathematical Modeling, Chaos, Computer Methods in Applied Mathematics and Engineering, Communications in Computational Physics, Computers and Fluids, Computer Physics Communications, International Journal of Computational Fluid Dynamics, IEEE Transactions on Plasma Science, Journal of Fluid Mechanics, Journal of Scientific Computing, Physics of Fluids, Physica D, International Journal of Numerical Methods in Fluids, Journal of Computational Physics, Journal of Nonlinear Science, Nature Communications, Nature Machine Intelligence, Nature Scientific Reports, Theoretical and Computational Fluid Dynamics, Atmospheric Science Letters, New Journal of Physics, Transactions of Machine Learning Research, Machine Learning Science and Technology, Neurips, ICLR.
\end{itemize}

\section*{Software developed}

\begin{enumerate}

\item R. Maulik, D. Fytanidis, S. Patel, B. Lusch, V. Vishwanath, PythonFoam: In-situ data analyses with OpenFOAM and Python. \url{https://github.com/argonne-lcf/PythonFOAM}.

\item R. Maulik, H. Sharma, S. Patel, B. Lusch, E. Jennings, TensorFlowFoam: A framework that enables the deployment of deep learning (in Python) and partial differential equation solutions concurrently in OpenFOAM - a C++-based open-source finite-volume based computational physics package. \url{https://github.com/argonne-lcf/TensorFlowFoam}.

\item R. Maulik, S. Pawar, PAR-RL: A framework that leverages the Ray library to deploy scalable deep reinforcement learning for arbitrary scientific environments on leadership class machines. Tested on ALCF supercomputer Theta for controlling simulations of dynamical systems. \url{https://github.com/Romit-Maulik/PAR-RL}.

\item R. Maulik, G. Mengaldo, PyParSVD: A Parallelized, streaming, and randomized implementation of the SVD for Python using mpi4py. \url{https://github.com/Romit-Maulik/PyParSVD}. DOI: 10.5281/zenodo.4562889.

\item G. Mengaldo, R. Maulik, PySPOD: Python Spectral Proper Orthogonal Decomposition. \url{https://github.com/mengaldo/PySPOD}.  DOI: https://doi.org/10.21105/joss.02862.

\end{enumerate}

% Footer
%\begin{center}
%  \begin{footnotesize}
%    Last updated: \today \\
%    \href{\footerlink}{\texttt{\footerlink}}
%  \end{footnotesize}
%\end{center}

\end{document}
